\anonsection{Перечень сокращений и условных обозначений}

\hypertarget{dc}{ЦОД --- центр обработки данных}

\hypertarget{soft}{ПО --- программное обеспечение}

\hypertarget{gnu}{GNU --- проект по разработке свободного программного обеспечения}

\hypertarget{gpl}{GPL --- General Public License, универсальная общественная лицензия}

\hypertarget{ram}{ОЗУ --- оперативное запоминающее устройство}

\hypertarget{ssd}{SSD --- Solid State Drive, твердотельный накопитель}

\hypertarget{nist}{NIST --- National Institute of Standards and Technology, Национальный институт стандартов и технологий}

\hypertarget{storage}{СХД --- система хранения данных}

\hypertarget{saas}{SaaS --- Software as a Service, программное обеспечение как услуга}

\hypertarget{paas}{PaaS --- Platform as a Service, платформа как услуга}

\hypertarget{iaas}{IaaS --- Infrastructure as a Service, инфраструктура как услуга}

\hypertarget{os}{ОС --- операционная система}

\hypertarget{soa}{SOA --- service-oriented architecture, сервис-ориентированная архитектура}

\hypertarget{aws}{AWS --- Amazon Web Services}

\hypertarget{gce}{GCE --- Google Compute Engine}

\hypertarget{omg}{OMG --- Object Management Group}

\hypertarget{it}{ИТ --- информационные технологии}

\hypertarget{csa}{CSA --- Cloud Security Alliance}

\hypertarget{dmtf}{DMTF --- Distributed Management Task Force}

\hypertarget{snia}{SNIA --- Storage Networking Industry Association}

\hypertarget{ogf}{OGF --- Open Grid Forum}

\hypertarget{occ}{OCC --- Open Cloud Consortium}

\hypertarget{oasis}{OASIS --- Organization for the Advancement of Structured Information Standards}

\hypertarget{ietf}{IETF --- Internet Engineering Task Force, инженерный совет Интернета}

\hypertarget{itu}{ITU --- International Telecommunications Union, Международный институт электросвязи}

\clearpage

\hypertarget{etsi}{ETSI --- European Telecommunications Standards Institute, Европейский институт телекоммуникационных стандартов}

\hypertarget{idps}{IDPS --- Intrusion Detection and Prevention Systems, руководство по системам обнаружения и предотвращения вторжений}

\hypertarget{ec2}{EC2 --- Elastic Compute Cloud, веб-сервис компании Amazon, предоставляющий вычислительные мощности в облаке}

\hypertarget{api}{API --- Application Programming Interface, интерфейс создания приложений}

\hypertarget{vcpu}{vCPU --- Virtual Central Processing Unit, виртуальное процессорное ядро}

\hypertarget{ami}{AMI --- Amazon Machine Images}

\hypertarget{ebs}{EBS --- Elastic Block Store, сервис постоянного хранилища блочного уровня для использования с инстансами Amazon}

\hypertarget{iot}{IoT --- Internet of Things, интернет вещей}

\hypertarget{ddos}{DDoS --- Distributed Denial of Service, распределенная атака на отказ}

\hypertarget{sdn}{SDN --- Software-defined Networking, программно-определяемая сеть}

\hypertarget{spi}{SPI --- Secutiry Parameter Index, индекс параметра обеспечения безопасности}

\hypertarget{bgp}{BGP --- Border Gateway Protocol, протокол граничного шлюза}

\hypertarget{asn}{ASN --- автономная сетевая система}

\hypertarget{dns}{DNS --- Domain Name System, система доменных имен}

\hypertarget{ntp}{NTP --- Network Time Protocol, протокол сетевого времени}

\hypertarget{snmp}{SNMP --- Simple Network Management Protocol, простой протокол сетевого управления}

\hypertarget{nda}{NDA --- Non-disclosure agreement, соглашение о неразглашении}

\hypertarget{ssh}{SSH --- Secure Shell, безопасная оболочка}

\hypertarget{sql}{SQL --- Structured Query Language, язык структурированных запросов}

\hypertarget{xss}{XSS --- Cross-Site Scripting, межсайтовый скриптинг}

\hypertarget{ftp}{FTP --- File Transfer Protocol, протокол передачи данных}

\hypertarget{ssl}{SSL --- Secure Sockets Layer, уровень защищенных cокетов}

\hypertarget{http}{HTTP --- HyperText Transfer Protocol, протокол передачи гипертекста}

\hypertarget{scp}{SCP --- Secure Copy Program}

\clearpage
