\section{Анализ полученных результатов}

В ходе обзора литературных источников по тематике исследования было рассмотрено понятие облачных вычислений и их развитие, рассмотрены классификации облачных услуг, проанализированы существующие стандарты безопасности и организации, принимающие участие в разработке этих стандартов.
Выполнен обзор наиболее известных поставщиков облачных услуг, рассмотрен российский рынок в период с 2014~г. по 2016~г.
Составлен список основных угроз облачной безопасности и исследованы тенденции развития облачных вычислений.

В ходе системного анализа была сформирована цель проектирования, разработан список функций проектируемой системы:
\begin{itemize}
  \item Ф1 --- авторизация и аутентификация пользователей;
  \item Ф2 --- сетевая защита;
  \item Ф3 --- идентификация и обработка инцидентов связанных с безопасностью;
  \item Ф4 --- предоставление доступа к услугам;
  \item Ф5 --- мониторинг.
\end{itemize}

Выделены подсистемы:
\begin{enumerate}
  \item подсистема аутентификации;
  \item подсистема авторизации;
  \item подсистема сетевой защиты;
  \item подсистема проверки целостности данных.
\end{enumerate}

Определена схема взаимодействия между подсистемами.

В соответствии с принципом функциональности составлена матрица инциденций функций системы и функций назначения подсистем.
Произведена декомпозиция подсистем, определены стороны развития системы, определены события и действия, некорректные с точки зрения правил функционирования системы.

В ходе вариантного анализа сравнивались альтернативы гипервизоров (KVM, Hyper-V, VMware vSphere) в соответствии с критериями цены, масштабируемости, отказоустойчивости и интерфейсов управления.
Построены матрицы парных сравнений второго и третьего уровня, исследованы согласованности матриц.
Синтезированы глобальные приоритеты альтернатив.
В результате анализа наибольшее предпочтение решено было отдать альтернативе В (VMware vSphere), однако для более точного определения гипервизора, необходимо сравнивать значительно большее число критериев.

В разделе описания облачной инфраструктуры представлены и описаны структурные схемы облачной инфраструктуры, а также архитектура системы безопасности.

В ходе экспериментальных исследований были проанализированы уязвимости 2016~г. в программном обеспечении, используемом в облачных вычислениях.
Исследованы основные ошибки в программном коде продуктов и способы их исправления.

Эксплуатирована уязвимость CVE-2016-5195, в ходе которой удалось получить права суперпользователя сервера, предложены способы защиты от уязвимостей в ядре Linux.

Для мониторинга уязвимостей была написана программа, анализирующая данные из открытого источника уязвимостей.
Данная программа может быть встроена в любую систему мониторинга и имеет возможность уведомлять системного администратора по Telegram.

\clearpage
