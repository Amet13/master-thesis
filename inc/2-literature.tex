\section{Обзор литературных источников по тематике исследования} \label{literature}

Исторически, ситуация сложилась так, что слово <<облако>> используется в качестве метафоры сети Интернет.
Позже, оно было использовано для изображения Интернет в компьютерных сетевых диаграммах и схемах.

Облачные вычисления можно обозначить, как выделение ресурсов в облаке.
В соответствии с \hyperlink{nist}{NIST} (Национальный институт стандартов и технологий), формальное определение облачных вычислений заключается в следующем:
<<Облачные вычисления являются моделью обеспечения повсеместного, удобного доступа по требованию по сети, общему пулу конфигурируемых вычислительных ресурсов (например сетей, серверов, систем хранения данных (\hyperlink{storage}{СХД}), приложений и услуг), которые могут быстро и с минимальными усилиями предоставлены для управления поставщиком услуг>> \cite{nist}.

Согласно опросам института Понемона в 2016 году, среди 3476 респондентов в сфере информационной безопасности из Соединенных Штатов Америки, Великобритании, Австралии, Германии, Японии, Франции, России, Индии и Бразилии, 73\% респондентов так или иначе используют облачные вычисления в своей инфраструктуре.
Особый рост внедрения облачных услуг произошел в последние 2 года \cite{gemalto}.

Хранение данных пользователей, почты и потребительских данных в облаке выросло в 2016 году по сравнению с 2014 годом (рис. \ref{cloud-data}).

\addimg{cloud-data}{1}{Сравнение использования облачных вычислений для хранения данных}{cloud-data}

Облачные вычисления являются результатом объединения большого количества технологий и связующего ПО для обеспечения ресурсов, необходимых для решения задачи, балансировки процессов, мониторинга, автоматизации и прочего.

Основными отличиями облачных услуг от классических являются (рис. \ref{cloud-tech}):
\begin{itemize}
  \item виртуализация;
  \item оркестратор;
  \item список услуг;
  \item портал самообслуживания;
  \item система тарификации и выставления счетов (биллинг).
\end{itemize}

\addimg{cloud-tech}{0.85}{Составные части облачных вычислений}{cloud-tech}

В вычислениях, виртуализация является процессом создания виртуальной (не физической) версии чего-либо, в том числе аппаратных платформ виртуального компьютера, операционных систем, устройств хранения данных и вычислительных ресурсов.

Виртуализация может быть предоставлена на различных аппаратных и программных уровнях, таких как центральный процессор, диск, память, файловые системы и прочее.
Чаще всего виртуализация используется для создания виртуальных машин и эмуляции различного оборудования для последующей установки операционных систем (\hyperlink{os}{ОС}) на них.

Виртуальные машины создаются на основе гипервизора, который работает поверх операционной системы хост-компьютера (физического компьютера, не виртуального).
С помощью гипервизора возможна эмуляция аппаратных средств, таких как процессор, диск, сеть, память, а также установка гостевых операционных систем на них.
Возможно создание нескольких гостевых виртуальных машин с различными операционными системами на гипервизоре.
Например, можно взять машину на Linux и установить ее на <<голое>> железо (bare-metal), и после настройки гипервизора возможно создание нескольких гостевых машин на Linux и Windows.

На данный момент все современные процессоры поддерживают аппаратную виртуализацию, это необходимо для безопасного и эффективного обмена ресурсами между хост-системой и гостевыми системами.
Большинство современных процессоров и гипервизоров также поддерживают вложенную виртуализацию, что позволяет создавать виртуальные машины внутри виртуальных машин.

Оркестратор является механизмом, выполняющий набор заданных операций по шаблону.
В сервис-ориентированной архитектуре (\hyperlink{soa}{SOA}), оркестровка сервисов реализуется согласно стандарту \hyperlink{bpel}{BPEL} (Business Process Execution Language).
Это позволяет автоматизировать процессы создания услуг пользователей в облачной среде.

Список услуг предоставляется пользователю в виде шаблонов готовых тарифов на портале самообслуживания, однако существуют и так называемые <<конфигураторы>>, которые позволяют пользователю создать шаблон индивидуально.

Портал самообслуживания является инструментом, с которым работает непосредственно пользователь.
Именно на портале обслуживания размещается список услуг, доступных клиенту.

Система тарификации и выставления счетов является необходимым механизмом для определения финансовых затрат пользователя в соответствии с затраченными ресурсами пользователя.

Провайдеры облачных услуг предлагают различные виды услуг, построенных поверх базового резервирования и освобождения ресурсов.
Большинство из этих услуг попадают в одну из следующих категорий (рис. \ref{aas}):
\begin{itemize}
  \item инфраструктура как услуга (IaaS);
  \item платформа как услуга (PaaS);
  \item программное обеспечение как услуга (SaaS).
\end{itemize}

\addimg{aas}{1}{Модели обслуживания облака}{aas}

Большинство провайдеров используют различные виды веб-интерфейса, на основе которого можно построить необходимый стек технологий.
Облачные провайдеры используют модель <<pay-as-you-go>>, в которой оплата производится только за время использования ресурсов.

Ключевыми функциями облачных вычислений являются:
\begin{itemize}
  \item скорость и масштабируемость, доступ к необходимым ресурсам можно получить одним щелчком мыши, что экономит время и обеспечивает гибкость, в зависимости от потребностей услуги, можно легко масштабировать ресурсы как вверх, так и вниз;
  \item стоимость, снижение первоначальных затрат на развертывание инфраструктуры позволяет сосредоточиться на приложениях и бизнесе, облачные провайдеры имеют возможность заранее оценить стоимость, что значительно облегчает планирование бюджета;
  \item легкий доступ к ресурсам, пользователи могут получить доступ к инфраструктуре из любого места и устройства, до тех пор, пока существует подключение к провайдеру;
  \item обслуживание, все работы по техническому обслуживанию ресурсов осуществляются поставщиком облачных услуг, пользователи не должны беспокоиться об этом;
  \item мультиаренда, несколько пользователей могут использовать один и тот же пул доступных ресурсов;
  \item надежность, ресурсы могут быть размещены в разных дата-центрах, для обеспечения повышенной надежности.
\end{itemize}

Как правило, облако может быть развернуто согласно следующим моделям (рис. \ref{clouds}):
\begin{itemize}
  \item частное облако;
  \item публичное облако;
  \item общественное облако;
  \item гибридное облако.
\end{itemize}

\addimg{clouds}{1}{Модели развертывания облака}{clouds}

Частное облако эксплуатируется исключительно одной организацией, оно может быть размещено внутри или снаружи сети организации и управляться внутренними командами или третьей стороной.
Частное облако можно построить с использованием такого программного обеспечения, как OpenStack.

Публичное облако доступно для всех пользователей, любой может использовать его после предоставления данных кредитной карты.
\hyperlink{aws}{AWS} (Amazon Web Services) и \hyperlink{gce}{GCE} (Google Compute Engine) являются примерами публичных облаков.

Гибридное облако является результатом объединения публичного и частных облаков.
Гибридное облако может быть использовано для хранения секретной информации о частном облаке, предлагая при этом услуги на основе этой информации из публичного облака.

Общественное облако, как правило, предназначено для сообщества или организации.

Поскольку технологии облачных вычислений относительно недавно начали свой путь к рынку массового потребления, одной из проблем обеспечения безопасности является отсутствие общепринятых стандартов в сфере предоставления облачных услуг.
Таким образом, в вопросах обеспечения безопасности, также не существует стандартов.
Данныя проблема все еще находится в процессе решения и развивается по трем основным направлениям.

Во-первых, облачные провайдеры создают собственные корпоративные стандарты, которые чаще всего публично не оглашаются.
В таком случае потребитель может полагаться исключительно на репутацию компании, предоставляющей облачные услуги.
Среди таких компаний можно выделить Google, Amazon, Microsoft, IBM, VMWare, Oracle и прочие.
Однако все же встречаются некоторые компании, такие как IBM, которые участвуют в открытии облачных стандартов.

Во-вторых, компании адаптируют свои услуги согласно существующим и устоявшимся стандартам безопасности (\hyperlink{giac}{GIAC}, \hyperlink{bsi}{BSI} и прочие), проходят соответствующие сертификации с последующим получием свидетельства.
Получение подобных сертификатов актуально в плане получения государственных и общественных заказов в долгосрочной перспективе \cite{itmo}.

В-третьих, различные правительственные, коммерческие и общественные организации принимают всяческие усилия по выработке требований к созданию безопасных облачных служб обработки информации.

Рабочая группа Object Management Group (\hyperlink{omg}{OMG}) в 2009 году была инициатором создания Cloud Standarts Summit.
Целью создания встречи является развитие информационных технологий (\hyperlink{it}{ИТ}) и согласование стандартов по проблемам государственных облачных сред.
В результате были созданы следующие рабочие группы:
\begin{itemize}
  \item Cloud Security Alliance (\hyperlink{csa}{CSA});
  \item Distributed Management Task Force (\hyperlink{dmtf}{DMTF});
  \item Storage Networking Industry Association (\hyperlink{snia}{SNIA});
  \item Open Grid Forum (\hyperlink{ogf}{OGF});
  \item Open Cloud Consortium (\hyperlink{occ}{OCC});
  \item Organization for the Advancement of Structured Information Standards (\hyperlink{oasis}{OASIS});
  \item TM Forum;
  \item Internet Engineering Task Force (\hyperlink{ietf}{IETF});
  \item International Telecommunications Union (\hyperlink{itu}{ITU});
  \item European Telecommunications Standards Institute (\hyperlink{etsi}{ETSI});
  \item National Institute of Standards and Technology (NIST);
  \item Object Management Group (OMG).
\end{itemize}

Наиболее известны достижения NIST, CSA, OASIS, а так же организации Open Data Center Alliance.

Cloud Security Alliance является некоммерческой организацией, созданной с целью продвижения идеи обеспечения безопасности облачных вычислений, а также для повышения уровня осведомленности по данной тематике как облачных поставщиков услуг, так и потребителей.
Ряд основных задач, выделаемых организацией CSA:
\begin{itemize}
  \item поддержка взаимоотношений потребителей и поставщиков услуг в требованиях безопасности и контроля качества;
  \item независимые исследования в части защиты;
  \item разработка и внедрение программ повышения осведомленности и обеспечению безопасности;
  \item разработка руководств и методических рекомендаций по обеспечению безопасности.
\end{itemize}

Руководство по безопасности критических областей в области облачных вычислений (Security Guidance for Critical Areas of Focus in Cloud Computing) покрывает основные аспекты и дает рекомендации потребителям облачных сред в тринадцати стратегически важных областях:
\begin{itemize}
  \item архитектурные решения сред облачных вычислений;
  \item государственное и корпоративное управление рисками;
  \item легальное и электронное открытие;
  \item соответствие техническим условиям и отчетность;
  \item управление жизненным циклом информации;
  \item портативность и совместимость;
  \item традиционная безопасность, непрерывность деятельности и восстановление в аварийных ситуациях;
  \item работа центра обработки данных;
  \item реакция на риски, уведомление и коррекционное обучение;
  \item прикладная безопасность;
  \item криптография и управление ключами;
  \item идентификация и управление доступом;
  \item виртуализация.
\end{itemize}

OASIS стимулирует развитие, сведение и принятие открытых стандартов для глобального информационного общества. Являясь источником многих современных основополагающих стандартов, организация видит облачные вычисления как естественное расширение сервисноориентированной архитектуры и моделей управления сетью \cite{psta}.
Технические агенты OASIS –– это набор участников, многие из которых активно участвуют в построении моделей облаков, профилей и расширений на существующие стандарты.
Примерами стандартов, разработанных в области политик безопасности, доступа и идентификации, являются OASIS SAML, XACML, SPML, WS-SecurityPolicy, WS-Trust, WS-Federation, KMIP и ORMS.

Организация Open Data Center Alliance объявила о публикации двух моделей использования (usage models), призванных снять наиболее значимые препятствия на пути внедрения облачных вычислений.
Первая модель использования называется <<The Provider Security Assurance>> (обеспечение безопасности на стороне провайдера).
В ней описаны требования к гранулированному описанию элементов обеспечения безопасности, которые должны предоставить поставщики услуг.

Вторая модель использования <<The Security Monitoring>> (Мониторинг соответствия требованиям безопасности) описывает требования к элементам, которые обеспечивают возможность мониторинга безопасности облачных услуг в реальном времени.
В совокупности две модели использования формируют набор требований, который может стать основой для создания стандартной модели обеспечения безопасности облачных услуг и осуществления мониторинга этих услуг в реальном времени.

Национальный Институт стандартов и технологий вместе с Американским национальным институтом стандартов (ANSI) участвует в разработке стандартов и спецификаций к программным решениям, используемым как в государственном секторе США, так и имеющим коммерческое применение.
Сотрудники NIST разрабатывают руководства, направленные на описание архитектуры облака, безопасность и стратегии использования, в числе которых руководство по системам обнаружения и предотвращения вторжений, руководство по безопасности и защите персональных данных при использовании публичных систем облачных вычислений.

В руководстве по системам обнаружения и предотвращения вторжений (NIST Guide to Intrusion Detection and Prevention Systems) даются характеристики технологий \hyperlink{idps}{IDPS} (Intrusion Detection and Prevention Systems) и рекомендации по их проектированию, внедрению, настройке, обслуживанию, мониторингу и поддержке.
Виды технологий IDPS различаются в основном по типам событий, за которыми проводится наблюдение, и по способам их применения.
Рассмотрены следующие четыре типа IDPS-технологий: сетевые, беспроводные, анализирующие поведение сети и централизованные.

В руководстве по безопасности и защите персональных данных при использовании публичных систем облачных вычислений (Guidelines on Security and Privacy in Public Cloud Computing) в том числе дается обзор проблем безопасности и конфиденциальности, имеющих отношение к среде облачных вычислений: обнаружение атак на гипервизор, цели атак, отдельно рассматриваются распределенные сетевые атаки.

Инфраструктура как услуга является одной из форм облачных вычислений, которая обеспечивает доступ по требованию к физическим и виртуальным вычислительным ресурсам, сети, межсетевым экранам, балансировщикам нагрузки и так далее.
Для обеспечения виртуальными вычислительными ресурсами, IaaS использует различные формы гипервизоров, таких как Xen, KVM, VMWare ESX/ESXi, Hyper-V и прочие.

Amazon Web Services является одним из лидеров в области предоставления услуг различных облачных сервисов.
С помощью Amazon Elastic Compute Cloud (\hyperlink{ec2}{EC2}), Amazon предоставляет клиентам IaaS-инфраструктуру (рис. \ref{ec2}).
Пользователь может управлять вычислительными ресурсами (инстансами) через веб-интерфейс Amazon EC2.
Существует возможность горизонтального и вертикального масштабирования ресурсов, в зависимости от требований.
AWS также предоставляет возможность управления инстансами посредством интерфейса командной строки и с помощью Application Programming Interface (\hyperlink{api}{API}).

\addimg{ec2}{1}{Пример инфраструктуры компании на основе EC2}{ec2}

В качестве гипервизора, Amazon EC2 использует Xen \cite{xen}.
Серввис предлагает инстансы различных конфигураций, которые можно выбрать в зависимости от требований.
Некоторые примеры различных конфигураций инстансов:
\begin{itemize}
  \item t2.nano: 512 Мб ОЗУ, 1 \hyperlink{vcpu}{vCPU} (виртуальных процессорных ядер), 32 или 64-битные платформы;
  \item c4.large: 4 Гб ОЗУ, 2 vCPU, 64-битная платформа;
  \item d2.8xlarge: 256 Гб ОЗУ, 36 vCPU, 64-битная платформа, 10G Ethernet.
\end{itemize}

Amazon EC2 предоставляет некоторые предварительно настроенные образы операционных систем, называемые Amazon Machine Images (\hyperlink{ami}{AMI}).
Эти образы могут быть использованы для быстрого запуска инстансов.
Пользователь также может создавать собственные образы ОС.
Amazon поддерживает настройки безопасности и доступа к сети для пользовательских инстансов.
С помощью Amazon Elastic Block Store (\hyperlink{ebs}{EBS}) пользователь может монтировать хранилища данных к инстансам.

Amazon EC2 имеет много других возможностей, что позволяет:
\begin{itemize}
  \item создавать <<гибкие>> IP-адреса для автоматического переназначения статического IP-адреса;
  \item предоставлять виртуальные частные облака;
  \item использовать услуги для мониторинга ресурсов и приложений;
  \item использовать автомасштабирование для динамического изменения доступных ресурсов.
\end{itemize}

Облачная платформа Azure, поддерживаемая компанией Microsoft, предлагает большой спектр облачных услуг, таких как: вычислительные мощности, платформы для мобильной и веб-разработки, хранилища данных, интернет вещей (\hyperlink{iot}{IoT}) и другие.

DigitalOcean позиционирует себя как простой облачный хостинг.
Все виртуальные машины (дроплеты) работают под управлением гипервизора KVM и используют SSD-накопители.
DigitalOcean предоставляет и другие функции, такие как IP-адреса расположенные в пределах одного дата-центра, частные сети, командные учетные записи и прочее.
Простота веб-интерфейса, высокое качество работы виртуальных машин и доступность для обычного пользователя способствовали быстрому росту компании.

На российском рынке облачного хостинга все еще наблюдается большой рост.
В связи с тем, что нет явно выраженного монополиста, таких как Amazon, Rackspace, Terramark, российский рынок очень разнообразный.
Конкуренция на рынке способствует значительному повышению качества предоставляемых услуг, а также гибкие тарифные планы и широкий перечень дополнительных услуг.

Также важную роль играет принятие Федерального закона от 21 июля 2014 г. № 242-ФЗ <<О внесении изменений в отдельные законодательные акты Российской Федерации в части уточнения порядка обработки персональных данных в информационно-телекоммуникационных сетях>> \cite{minsvyaz}.
Крупные компании обязаны хранить персональные данные пользователй на территории России, что способствует консолидации российского рынка облачных услуг.

Крупнейшие поставщики услуг ЦОД 2016 г. \cite{cnews} представлены в табл. \ref{dc-table}.
\begin{table}[H]
  \caption{Крупнейшие поставщики услуг ЦОД в 2016 году}\label{dc-table}
  \begin{tabular}{|p{0.6cm}|p{2.6cm}|p{3cm}|p{3.5cm}|p{3.5cm}|}
  \hline \# & Название компании & Количество доступных стойко-мест & Количество размещенных стойко-мест & Загруженность мощностей (\%) \\
  \hline 1 & Ростелеком & 3 900 & 3 432 & 88 \\
  \hline 2 & DataLine & 3 703 & 2 988 & 81 \\
  \hline 3 & DataPro & 3 000 & н/д & н/д \\
  \hline 4 & Linxtelecom & 2 040 & н/д & н/д \\
  \hline 5 & Selectel & 1 500 & 1 200 & 80 \\
  \hline 6 & Stack Group & 1 400 & 854 & 61 \\
  \hline 7 & Ай-Теко & 1 200 & 960 & 80 \\
  \hline 8 & DataSpace & 1 152 & 820 & 71 \\
  \hline 9 & SDN & 1 074 & 815 & 76 \\
  \hline 10 & Крок & 1 000 & 980 & 98 \\
  \hline
  \end{tabular}
\end{table}

Данные DataLine включают показатели 7 ЦОД, расположенных на площадказ OST и NORD.
Данные по количеству введенных в эксплуатацию и реально размещенных стоек в ЦОД DataPro и Linxtelecom отсутствуют.

По итогам 2015 г. CNews Analytics впервые составил рейтинг крупнейших поставщиков IaaS.
В исследовании приняли участие 14 компаний, совокупная выручка которых составила 3,8 млрд. рублей.
По сравнению с 2014 г. участники заработали на 63\% больше.
Все участники рейтинга продемонстрировали положительную динамику за исключением компании Inoventica (-3\%).
Высокие темпы роста свидетельствуют о том, что рынок IaaS находится в начале своего становления.
Многие участники рейтинга вышли на этот рынок только в 2014-2015 г., чем объяснятся наличие большого числа компаний с ростом более в чем 3 раза: StackGroup (+733\%), 1cloud.ru (+911\%), CaravanAero (+1220\%).

Сравнение крупнейших поставщиков IaaS в 2016 г. \cite{cnews} представлено в табл. \ref{iaas-table}.
\begin{table}[H]
  \caption{Крупнейшие поставщики IaaS в 2016 году}\label{iaas-table}
  \begin{tabular}{|p{0.5cm}|p{2.5cm}|p{3.5cm}|p{3.5cm}|p{4.5cm}|}
  \hline \# & Название компании & Выручка IaaS в 2015 г. (тыс.р.) & Выручка IaaS в 2014 г. (тыс.р.) & ЦОД \\
  \hline 1 & ИТ-Град & 857 245 & 358 680 & Datalahti, DataSpace, SDN, AHOST \\
  \hline 2 & Крок & 667 609 & 440 315 & Волочаевская-1/2, Компрессор \\
  \hline 3 & Ай-Теко & 618 500 & 565 800 & ТрастИнфо \\
  \hline 4 & DataLine & 500 960 & 358 400 & NORD1/2/3/4, OST1/2/3 \\
  \hline 5 & SoftLine & 367 000 & 152 000 & н/д \\
  \hline 6 & Cloud4Y & 304 600 & 267 400 & Цветочная, М8/9/10, Nord, Equinix FR5, EvoSwitch \\
  \hline 7 & Stack Group & 182 900 & 21 948 & M1 \\
  \hline 8 & ActiveCloud & 103 702 & 56 910 & DataLine \\
  \hline 9 & Inoventica & 79 000 & 81 000 & н/д \\
  \hline 10 & 1cloud.ru & 78 307 & 7 743 & SDN, DataSpace \\
  \hline
  \end{tabular}
\end{table}

Почти все участники рейтинга SaaS продемонстрировали положительную динамику выручки, при этом у 10 компаний оборот вырос более чем на 50\%, а четыре облачных провайдера зафиксировали рост выручки более чем на 100\%: Naumen (+358\%), amoCRM (+159\%), ИТ-Град (+134\%) и Artsofte (+125\%).

Сравнение крупнейших поставщиков SaaS в 2016 г. \cite{cnews} представлено в табл. \ref{saas-table}.
\begin{table}[H]
  \caption{Крупнейшие поставщики SaaS в 2016 году}\label{saas-table}
  \begin{tabular}{|p{0.5cm}|p{3.5cm}|p{3.5cm}|p{3.5cm}|p{3.5cm}|}
  \hline \# & Название компании & Выручка SaaS в 2015 г. (тыс.р.) & Выручка SaaS в 2014 г. (тыс.р.) & Рост выручки 2015/2014 (\%) \\
  \hline 1 & СКБ Контур & 6 970 000 & 5 500 000 & 27 \\
  \hline 2 & Манго Телеком & 1 808 000 & 1 350 000 & 34 \\
  \hline 3 & B2B-Center & 1 163 300 & 1 155 842 & 1 \\
  \hline 4 & Барс Груп & 1 074 000 & 910 000 & 18 \\
  \hline 5 & SoftLine & 1 034 000 & 636 000 & 63 \\
  \hline 6 & Корпус Консалтинг СНГ & 783 511 & 602 825 & 32 \\
  \hline 7 & Terrasoft & 657 654 & 476 561 & 38 \\
  \hline 8 & Телфин & 398 500 & 317 900 & 25 \\
  \hline 9 & МойСклад & 395 000 & 265 000 & 49 \\
  \hline 10 & ИТ-Град & 265 400 & 113 420 & 134 \\
  \hline
  \end{tabular}
\end{table}

Все центры обработки данных сталкиваются с проблемой вывода тепловой энергии, являющейся побочным теплом от плотно укомплектованных серверных стоек.
Общий экономический и экологический эффект высокого потребления электроэнергии подогревает интерес к так называемым <<зеленым>> ЦОД \cite{cnewsdc}.

Электроэнергия является одной из главных расходных статей в проектировании ЦОД, таких крупных компаний как Google, eBay, Microsoft, Amazon и прочих.

В 2008 году компания McKinsey \& Company провела аналитический обзор выбросов углекислого газа от выработки электроэнергии для нужд ЦОД \cite{greendc}.
Согласно исследованиям, суммарные показатели выброса углекислого газа ЦОД равны выбросам такой страны как Аргентина.

<<Озеленение>> дата-центров может происходить за счет использования альтернативных возобновляемых источников энергии (солнце, ветер).
В мире уже имеется ряд ЦОД, успешно использующих этот подход.
Первый дата-центр в США, который полностью обеспечивает мощности для своих серверов за счет энергии ветра от рядом расположенной турбины расположен в штате Иллинойс.

Компания Other World Computing в 2009 году начала использовать 40-метровую ветровую турбину для обеспечения всей электрической мощности для своего здания в Вудстоке, где находится штаб-квартира компании и ЦОД.

Сервис Microsoft Virtual Earth управляется вне дата-центра, в контейнере, расположенном в штате Колорадо, который полностью питается возобновляемой энергией ветра.
Microsoft использует выгодное преимущество использования контейнеров: их легко можно поместить рядом с источником возобновляемой энергии, что позволяет компании и сокращать выбросы углекислого газа.

Все большее количество дата-центров направляет тепло из своих помещений в находящиеся неподалеку дома, офисы, оранжереи, бассейны.
Возможность заново использовать излишнее тепло от серверов предусматривается на этапе разработки нового дата-центра, помогая повысить энергетическую эффективность оборудования.

Компания Microsoft в рамках Project Natick создала прототип ЦОД под названием Leona Philpot.
Прототип серверной фермы был погружен в километре от тихоокеанского побережья США и успешно эксплуатировался на протяжении четырех месяцев.
Leona Philpot был снабжен большим количеством теплообменников, которые в свою очередь передавали излишнее тепло от серверов в холодную воду.

Инженеры Microsoft утверждают, что такой способ оптимизации температурного режима серверного оборудования является одним из наиболее эффективных и в то же время дешевых методов охлаждения серверных ферм.

Помимо <<зеленых>> ЦОД одним из трендов последних лет является Software-defined Networking (\hyperlink{sdn}{SDN}).
Основными трендами развития корпоративных сетей и сетей центров обработки данных являются:
\begin{itemize}
  \item рост объемов трафика и изменение его структуры;
  \item рост числа пользователей мобильных приложений и социальных сетей;
  \item высокопроизводительные кластеры для обработки большого количества данных (Big Data);
  \item виртуализация для предоставления облачных услуг.
\end{itemize}

Программно-конфигурируемая сеть --- сеть передачи данных, в которой уровень управления сетью отделен от устройств передачи данных и реализуется программно, одна из форм виртуализации вычислительных ресурсов (рис. \ref{vnet}).

\addimg{vnet}{1}{Схема программно-конфигурируемой сети}{vnet}

Если рассмотреть современный маршрутизатор или коммутатор, то он логически состоит из трех компонентов:
\begin{itemize}
  \item уровень управления --- это командный интерфейс, встроенный веб-сервер или API и протоколы управления, задача этого уровня обеспечить управляемость устройством;
  \item уровень управления трафиком --- это различные алгоритмы и функционал задачей которого является автоматическая реакция на изменения трафика;
  \item передача трафика --- функционал обеспечивающий физическую передачу данных, уровень микросхем и сетевых пакетов.
\end{itemize}

OpenFlow является стандартным протоколом, использующийся в построении SDN.
Коммутаторы с поддержкой OpenFlow выпускает компания Hewlett-Packard.
Компания считает, что SDN должна строиться на базе открытых стандартах, чтобы каждый желающий мог в этом поучаствовать.
Такая открытая экосистема позволит возобновить процесс внедрения инноваций в области сетевых технологий.

\clearpage
