\section{Обзор литературных источников по тематике исследования}

Исторически, ситуация сложилась так, что слово <<облако>> используется в качестве метафоры сети Интернет.
Позже, оно было использовано для изображения Интернет в компьютерных сетевых диаграммах и схемах.

\subsection{Становление и развитие облачных вычислений}

Облачные вычисления можно обозначить, как выделение ресурсов по требованию.
В соответствии с Национальным институтом стандартов и технологий (\hyperlink{nist}{NIST}), формальное определение облачных вычислений заключается в следующем:
<<Облачные вычисления являются моделью обеспечения повсеместного, удобного доступа по требованию по сети, общему пулу конфигурируемых вычислительных ресурсов (например сетей, серверов, систем хранения данных (\hyperlink{storage}{СХД}), приложений и услуг), которые могут быстро и с минимальными усилиями предоставлены для управления поставщиком услуг>> \cite{nist}.

Согласно опросам института Понемон в 2016~г., среди 3476 респондентов в сфере информационной безопасности из Соединенных Штатов Америки, Великобритании, Австралии, Германии, Японии, Франции, России, Индии и Бразилии, 73\% респондентов так или иначе используют облачные вычисления в своей инфраструктуре.
Особый рост внедрения облачных услуг произошел в последние 2 года \cite{gemalto}.

Хранение данных пользователей, почты и потребительских данных в облаке выросло в 2016~г. по сравнению с 2014~г. (рис. \ref{cloud-data}).

\addimg{cloud-data}{1}{Использование облака для хранения данных}{cloud-data}

Облачные вычисления являются результатом объединения большого количества технологий и связующего ПО для обеспечения ресурсов, необходимых для решения задачи, балансировки процессов, мониторинга, автоматизации и прочего.

Основными отличиями предоставления облачных услуг от <<классических>> является использование:
\begin{itemize}
  \item виртуализации;
  \item оркестратора;
  \item списка (каталога) услуг;
  \item портала самообслуживания;
  \item системы тарификации и выставления счетов (биллинга).
\end{itemize}

В вычислениях, виртуализация является процессом создания виртуальной версии чего-либо, в том числе аппаратных платформ виртуального компьютера, операционных систем (\hyperlink{os}{ОС}), устройств хранения данных и вычислительных ресурсов.

Виртуализация может быть предоставлена на различных аппаратных и программных уровнях, таких как центральный процессор, диск, память, файловые системы и прочее.
Чаще всего виртуализация используется для создания виртуальных машин и эмуляции различного оборудования для последующей установки операционных систем на них.

Виртуальные машины создаются на основе гипервизора, который работает поверх операционной системы хост-компьютера (физического компьютера, не виртуального).
С помощью гипервизора возможна эмуляция аппаратных средств, таких как процессор, диск, сеть, память, а также установка гостевых операционных систем на них.
Возможно создание нескольких гостевых виртуальных машин с различными операционными системами на гипервизоре.
Например, можно взять машину на Linux и установить ее на <<голое>> железо (bare-metal), и после настройки гипервизора возможно создание нескольких гостевых машин на Linux и Windows.

На данный момент все современные процессоры поддерживают аппаратную виртуализацию, это необходимо для безопасного и эффективного обмена ресурсами между хост-системой и гостевыми системами.
Большинство современных процессоров и гипервизоров также поддерживают вложенную виртуализацию, что позволяет создавать виртуальные машины друг внутри друга.

Оркестратор является механизмом, выполняющим набор заданных операций по шаблону.
В сервис-ориентированной архитектуре (\hyperlink{soa}{SOA}), оркестровка сервисов реализуется согласно стандарту BPEL.
Это позволяет автоматизировать процессы создания услуг пользователей в облачной среде.

Список услуг предоставляется пользователю в виде шаблонов готовых тарифов на портале самообслуживания, однако существуют и <<конфигураторы>>, которые позволяют пользователю создать шаблон индивидуально.

Портал самообслуживания является инструментом, с которым работает непосредственно пользователь.
Именно на портале обслуживания размещается список услуг, доступных пользователю.

Система тарификации и выставления счетов является необходимым механизмом для определения финансовых затрат пользователя в соответствии с затраченными ресурсами пользователя.

\subsection{Классификация облачных услуг}

Поставщики облачных услуг (провайдеры) предлагают различные виды услуг, построенные поверх базового резервирования и освобождения ресурсов.
Большинство из этих услуг попадают в одну из следующих категорий (рис. \ref{aas}):
\begin{itemize}
  \item инфраструктура как услуга (\hyperlink{iaas}{IaaS});
  \item платформа как услуга (\hyperlink{paas}{PaaS});
  \item программное обеспечение как услуга (\hyperlink{saas}{SaaS}).
\end{itemize}

\addimg{aas}{1}{Модели обслуживания облака}{aas}

Большинство поставщиков облачных услуг используют различные виды программных интерфейсов, в том числе и веб-интерфейсов, на основе которых можно построить необходимый стек технологий.
Облачные провайдеры используют модель <<pay-as-you-go>>, в которой оплата производится только за время использования ресурсов.

Ключевыми функциями облачных вычислений являются:
\begin{itemize}
  \item высокая скорость работы и гибкая масштабируемость;
  \item низкая стоимость услуг;
  \item легкий доступ к ресурсам;
  \item отсутствие необходимости в обслуживании оборудования;
  \item возможность совместного использования;
  \item надежность.
\end{itemize}

Доступ к необходимым ресурсам можно получить одним щелчком мыши, что экономит время и обеспечивает гибкость.
В зависимости от потребностей услуги, можно легко масштабировать ресурсы как горизонтально, так и вертикально.
Снижение первоначальных затрат на развертывание инфраструктуры позволяет сосредоточиться на приложениях и бизнесе.
Компания имеет возможность заранее оценить стоимость, что значительно облегчает планирование бюджета.
Пользователи могут получить доступ к инфраструктуре из любого места и устройства, до тех пор, пока существует интернет-подключение к поставщику.
Все работы по техническому обслуживанию ресурсов осуществляются поставщиком облачных услуг, пользователи не должны беспокоиться об этом.
Несколько пользователей могут использовать один и тот же пул доступных ресурсов.
Ресурсы могут быть размещены в разных дата-центрах, для обеспечения повышенной надежности.

Широкое применение облачных вычислений позволяет использовать различные сценарии его использования.
Как правило, облако может быть развернуто согласно следующим моделям (рис. \ref{clouds}):
\begin{itemize}
  \item частное облако;
  \item публичное облако;
  \item общественное облако;
  \item гибридное облако.
\end{itemize}

\addimg{clouds}{1}{Модели развертывания облака}{clouds}

Частное облако эксплуатируется исключительно одной организацией, оно может быть размещено внутри или снаружи сети организации и управляться внутренними командами или третьей стороной.
Частное облако можно построить с использованием такого программного обеспечения, как OpenStack.

Публичное облако доступно для всех пользователей, любой может использовать его после предоставления платежных данных.
Amazon Web Services (\hyperlink{aws}{AWS}) и Google Compute Engine (\hyperlink{gce}{GCE}) являются примерами публичных облаков.

Гибридное облако является результатом объединения публичных и частных облаков.
Гибридное облако может быть использовано для хранения секретной информации о частном облаке, предлагая при этом услуги на основе этой информации из публичного.

Общественное облако, как правило, предназначено для сообщества или организации.

\subsection{Стандартизация в области облачных вычислений}

Облачные технологии относительно недавно вышли на рынок массового потребления, поэтому пока еще не существует четких общепринятых стандартов в сфере предоставления облачных услуг и обеспечении безопасности.
Поставщики облачных услуг обходят эту проблему тремя путями.

Во-первых, облачные провайдеры создают собственные корпоративные стандарты, которые чаще всего публично не оглашаются.
В таком случае потребитель может полагаться исключительно на репутацию компании, предоставляющей облачные услуги.
Среди таких компаний можно выделить Google, Amazon, Microsoft, IBM, VMware, Oracle и прочие.
Однако иногда такие как компании как IBM, участвуют в открытии облачных стандартов.

Во-вторых, компании адаптируют свои услуги согласно существующим и устоявшимся стандартам безопасности, проходят соответствующие сертификации с последующим получением свидетельства.
Получение подобных сертификатов актуально в плане получения государственных и общественных заказов в долгосрочной перспективе \cite{itmo}.

В-третьих, различные правительственные, коммерческие и общественные организации принимают всяческие усилия по выработке требований к созданию безопасных облачных служб обработки информации.

Рабочая группа Object Management Group (\hyperlink{omg}{OMG}) в 2009~г. была инициатором создания Cloud Standarts Summit.
Целью создания встречи является развитие информационных технологий (\hyperlink{it}{ИТ}) и согласование стандартов по проблемам государственных облачных сред.
В результате были созданы следующие рабочие группы:
\begin{itemize}
  \item Cloud Security Alliance (\hyperlink{csa}{CSA});
  \item Distributed Management Task Force (\hyperlink{dmtf}{DMTF});
  \item Storage Networking Industry Association (\hyperlink{snia}{SNIA});
  \item Open Grid Forum (\hyperlink{ogf}{OGF});
  \item Open Cloud Consortium (\hyperlink{occ}{OCC});
  \item Organization for the Advancement of Structured Information Standards (\hyperlink{oasis}{OASIS});
  \item TM Forum;
  \item Internet Engineering Task Force (\hyperlink{ietf}{IETF});
  \item International Telecommunications Union (\hyperlink{itu}{ITU});
  \item European Telecommunications Standards Institute (\hyperlink{etsi}{ETSI});
  \item National Institute of Standards and Technology (NIST);
  \item Object Management Group (OMG).
\end{itemize}

Наиболее известны достижения организаций NIST, CSA, OASIS, Open Data Center Alliance.

Cloud Security Alliance является некоммерческой организацией, созданной с целью продвижения идей обеспечения безопасности облачных вычислений, а также для повышения уровня осведомленности по данной тематике как поставщиков облачных услуг, так и потребителей.
Ряд основных задач, выделяемых организацией CSA:
\begin{itemize}
  \item поддержка взаимоотношений потребителей и поставщиков услуг в требованиях безопасности и контроля качества;
  \item независимые исследования по части защиты;
  \item разработка и внедрение программ повышения осведомленности и обеспечению безопасности;
  \item разработка руководств и методических рекомендаций по обеспечению безопасности.
\end{itemize}

Руководство по безопасности критических областей в области облачных вычислений (Security Guidance for Critical Areas of Focus in Cloud Computing) покрывает основные аспекты и дает рекомендации потребителям облачных сред в тринадцати стратегически важных областях:
\begin{itemize}
  \item архитектурные решения сред облачных вычислений;
  \item государственное и корпоративное управление рисками;
  \item легальное и электронное открытие;
  \item соответствие техническим условиям и отчетность;
  \item управление жизненным циклом информации;
  \item портативность и совместимость;
  \item традиционная безопасность, непрерывность деятельности и восстановление в аварийных ситуациях;
  \item работа центра обработки данных;
  \item реакция на риски, уведомление и коррекционное обучение;
  \item прикладная безопасность;
  \item криптография и управление ключами;
  \item идентификация и управление доступом;
  \item виртуализация.
\end{itemize}

OASIS стимулирует развитие, сведение и принятие открытых стандартов для глобального информационного общества. Являясь источником многих современных основополагающих стандартов, организация видит облачные вычисления как естественное расширение сервис-ориентированной архитектуры и моделей управления сетью \cite{psta}.
Технические агенты OASIS –– это набор участников, многие из которых активно участвуют в построении моделей облаков, профилей и расширений на существующие стандарты.
Примерами стандартов, разработанных в области политик безопасности, доступа и идентификации, являются OASIS SAML, XACML, SPML, WS-SecurityPolicy, WS-Trust, WS-Federation, KMIP и ORMS.

Организация Open Data Center Alliance объявила о публикации двух моделей использования, призванных снять наиболее значимые препятствия на пути внедрения облачных вычислений.
Первая модель использования называется <<The Provider Security Assurance>>.
В ней описаны требования к гранулированному описанию элементов обеспечения безопасности, которые должны предоставить поставщики услуг.

Вторая модель использования <<The Security Monitoring>> описывает требования к элементам, которые обеспечивают возможность мониторинга безопасности облачных услуг в реальном времени.
В совокупности две модели использования формируют набор требований, который может стать основой для создания стандартной модели обеспечения безопасности облачных услуг и осуществления мониторинга этих услуг в реальном времени.

Национальный институт стандартов и технологий вместе с Американским национальным институтом стандартов (\hyperlink{ansi}{ANSI}) участвует в разработке стандартов и спецификаций к программным решениям, используемым как в государственном секторе США, так и имеющим коммерческое применение.
Сотрудники NIST разрабатывают руководства, направленные на описание облачной архитектуры, безопасность и стратегии использования, в числе которых руководство по системам обнаружения и предотвращения вторжений, руководство по безопасности и защите персональных данных при использовании публичных систем облачных вычислений.

В руководстве по системам обнаружения и предотвращения вторжений даются характеристики технологий \hyperlink{idps}{IDPS} и рекомендации по их проектированию, внедрению, настройке, обслуживанию, мониторингу и поддержке.
Виды технологий IDPS различаются в основном по типам событий, за которыми проводится наблюдение, и по способам их применения.
Рассмотрены следующие четыре типа IDPS-технологий: сетевые, беспроводные, анализирующие поведение сети и централизованные.

В руководстве по безопасности и защите персональных данных при использовании публичных систем облачных вычислений в том числе дается обзор проблем безопасности и конфиденциальности, имеющих отношение к среде облачных вычислений: обнаружение атак на гипервизор, цели атак, отдельно рассматриваются распределенные сетевые атаки.

Инфраструктура как услуга является одной из форм облачных вычислений, которая обеспечивает доступ по требованию к физическим и виртуальным вычислительным ресурсам, сетям, межсетевым экранам, балансировщикам нагрузки и так далее.
Для организации виртуальных вычислительных ресурсов, IaaS использует различные формы гипервизоров, таких как Xen, KVM, VMware ESX/ESXi, Hyper-V и прочие (рис. \ref{gartnerv}).

\addimg{gartnerv}{0.6}{Квадрат Gartner для систем виртуализации}{gartnerv}

\subsection{Обзор зарубежных облачных провайдеров}

Amazon Web Services является одним из лидеров в области предоставления услуг различных облачных сервисов (рис. \ref{gartnerp}).

\addimg{gartnerp}{0.6}{Квадрат Gartner для облачных провайдеров}{gartnerp}

С помощью Amazon Elastic Compute Cloud (\hyperlink{ec2}{EC2}), Amazon предоставляет пользователям IaaS-инфраструктуру.
Пользователь может управлять вычислительными ресурсами (инстансами) через веб-интерфейс Amazon EC2.
Существует возможность горизонтального и вертикального масштабирования ресурсов, в зависимости от требований.
AWS также предоставляет возможность управления инстансами посредством интерфейса командной строки и с помощью Application Programming Interface (\hyperlink{api}{API}).

В качестве гипервизора, Amazon EC2 использует Xen \cite{xen}.
Сервис предлагает инстансы различных конфигураций, которые можно выбрать в зависимости от требований.
Некоторые примеры конфигураций виртуальных машин:
\begin{itemize}
  \item t2.nano: 512 Мб ОЗУ, 1 \hyperlink{vcpu}{vCPU} (виртуальных процессорных ядер), 32 или 64-битные платформы;
  \item c4.large: 4 Гб ОЗУ, 2 vCPU, 64-битная платформа;
  \item d2.8xlarge: 256 Гб ОЗУ, 36 vCPU, 64-битная платформа, 10G Ethernet.
\end{itemize}

Amazon EC2 предоставляет некоторые предварительно настроенные образы операционных систем, называемые Amazon Machine Images (\hyperlink{ami}{AMI}).
Эти образы могут быть использованы для быстрого запуска виртуальных машин.
Пользователь также может создавать собственные образы ОС.
Amazon поддерживает настройки безопасности и доступа к сети для пользовательских виртуальных машин.
С помощью Amazon Elastic Block Store (\hyperlink{ebs}{EBS}) пользователь может монтировать хранилища данных к инстансам.

Amazon EC2 имеет много других возможностей, что позволяет:
\begin{itemize}
  \item создавать <<гибкие>> IP-адреса для автоматического переназначения статического IP-адреса;
  \item предоставлять виртуальные частные облака;
  \item использовать услуги для мониторинга ресурсов и приложений;
  \item использовать автомасштабирование для динамического изменения доступных ресурсов.
\end{itemize}

Облачная платформа Azure, поддерживаемая компанией Microsoft, предлагает большой спектр облачных услуг, таких как: вычислительные мощности, платформы для мобильной и веб-разработки, хранилища данных, интернет вещей (\hyperlink{iot}{IoT}) и другие.

DigitalOcean позиционирует себя как простой облачный хостинг.
Все виртуальные машины (дроплеты) работают под управлением гипервизора KVM и используют SSD-накопители.
DigitalOcean предоставляет и другие функции, такие как IP-адреса расположенные в пределах одного дата-центра, частные сети, командные учетные записи и прочее.
Простота веб-интерфейса, высокое качество работы виртуальных машин и доступность для обычного пользователя способствовали быстрому росту компании.

\subsection{Развитие российского рынка облачных услуг}

На российском рынке облачного хостинга все еще наблюдается большой рост.
В связи с тем, что нет явно выраженного монополиста, таких как Amazon или Rackspace, российский рынок очень разнообразный.
Конкуренция на рынке способствует значительному повышению качества предоставляемых услуг, а также гибкие тарифные планы и широкий перечень дополнительных услуг.

Также важную роль играет принятие Федерального закона от 21 июля 2014~г. № 242-ФЗ <<О внесении изменений в отдельные законодательные акты Российской Федерации в части уточнения порядка обработки персональных данных в информационно-телекоммуникационных сетях>> \cite{minsvyaz}.
Крупные компании обязаны хранить персональные данные пользователей на территории России, что способствует консолидации российского рынка облачных услуг.

Крупнейшие поставщики услуг ЦОД 2016~г. \cite{cnews} представлены в табл. \ref{dc-table}.
\begin{table}[H]
  \caption{Крупнейшие поставщики услуг ЦОД в 2016~г.}\label{dc-table}
  \begin{tabular}{|p{0.6cm}|p{2.6cm}|p{3cm}|p{3.5cm}|p{3.5cm}|}
  \hline \# & Название компании & Количество доступных стойко-мест & Количество размещенных стойко-мест & Загруженность мощностей (\%) \\
  \hline 1 & Ростелеком & 3 900 & 3 432 & 88 \\
  \hline 2 & DataLine & 3 703 & 2 988 & 81 \\
  \hline 3 & DataPro & 3 000 & н/д & н/д \\
  \hline 4 & Linxtelecom & 2 040 & н/д & н/д \\
  \hline 5 & Selectel & 1 500 & 1 200 & 80 \\
  \hline 6 & Stack Group & 1 400 & 854 & 61 \\
  \hline 7 & Ай-Теко & 1 200 & 960 & 80 \\
  \hline 8 & DataSpace & 1 152 & 820 & 71 \\
  \hline 9 & SDN & 1 074 & 815 & 76 \\
  \hline 10 & Крок & 1 000 & 980 & 98 \\
  \hline
  \end{tabular}
\end{table}

Данные DataLine включают показатели 7 ЦОД, расположенных на площадках OST и NORD.
Данные по количеству введенных в эксплуатацию и реально размещенных стоек в ЦОД DataPro и Linxtelecom отсутствуют.

По итогам 2015~г. CNews Analytics впервые составил рейтинг крупнейших поставщиков IaaS.
В исследовании приняли участие 14 компаний, совокупная выручка которых составила 3,8 млрд. рублей.
По сравнению с 2014~г. участники заработали на 63\% больше.
Все участники рейтинга продемонстрировали положительную динамику за исключением компании Inoventica (-3\%).
Высокие темпы роста свидетельствуют о том, что рынок IaaS находится в начале своего становления.
Многие участники рейтинга вышли на этот рынок только в 2014-2015~г., чем объяснятся наличие большого числа компаний с ростом более в чем 3 раза: StackGroup (+733\%), 1cloud.ru (+911\%), CaravanAero (+1220\%).

Сравнение крупнейших поставщиков IaaS в 2016~г. \cite{cnews} представлено в табл. \ref{iaas-table}.
\begin{table}[H]
  \caption{Крупнейшие поставщики IaaS в 2016~г.}\label{iaas-table}
  \begin{tabular}{|p{0.5cm}|p{2.5cm}|p{3.5cm}|p{3.5cm}|p{4.5cm}|}
  \hline \# & Название компании & Выручка IaaS в 2015~г. (тыс.р.) & Выручка IaaS в 2014~г. (тыс.р.) & ЦОД \\
  \hline 1 & ИТ-Град & 857 245 & 358 680 & Datalahti, DataSpace, SDN, AHOST \\
  \hline 2 & Крок & 667 609 & 440 315 & Волочаевская-1/2, Компрессор \\
  \hline 3 & Ай-Теко & 618 500 & 565 800 & ТрастИнфо \\
  \hline 4 & DataLine & 500 960 & 358 400 & NORD1/2/3/4, OST1/2/3 \\
  \hline 5 & SoftLine & 367 000 & 152 000 & н/д \\
  \hline 6 & Cloud4Y & 304 600 & 267 400 & Цветочная, М8/9/10, Nord, Equinix FR5, EvoSwitch \\
  \hline 7 & Stack Group & 182 900 & 21 948 & M1 \\
  \hline 8 & ActiveCloud & 103 702 & 56 910 & DataLine \\
  \hline 9 & Inoventica & 79 000 & 81 000 & н/д \\
  \hline 10 & 1cloud.ru & 78 307 & 7 743 & SDN, DataSpace \\
  \hline
  \end{tabular}
\end{table}

Почти все участники рейтинга SaaS продемонстрировали положительную динамику выручки, при этом у 10 компаний оборот вырос более чем на 50\%, а четыре поставщика облачных услуг зафиксировали рост выручки более чем на 100\%: Naumen (+358\%), amoCRM (+159\%), ИТ-Град (+134\%) и Artsofte (+125\%).

Сравнение крупнейших поставщиков SaaS в 2016~г. \cite{cnews} представлено в табл. \ref{saas-table}.
\begin{table}[H]
  \caption{Крупнейшие поставщики SaaS в 2016~г.}\label{saas-table}
  \begin{tabular}{|p{0.5cm}|p{3.5cm}|p{3.5cm}|p{3.5cm}|p{3.5cm}|}
  \hline \# & Название компании & Выручка SaaS в 2015~г. (тыс.р.) & Выручка SaaS в 2014~г. (тыс.р.) & Рост выручки 2015/2014~г. (\%) \\
  \hline 1 & СКБ Контур & 6 970 000 & 5 500 000 & 27 \\
  \hline 2 & Манго Телеком & 1 808 000 & 1 350 000 & 34 \\
  \hline 3 & B2B-Center & 1 163 300 & 1 155 842 & 1 \\
  \hline 4 & Барс Груп & 1 074 000 & 910 000 & 18 \\
  \hline 5 & SoftLine & 1 034 000 & 636 000 & 63 \\
  \hline 6 & Корпус Консалтинг СНГ & 783 511 & 602 825 & 32 \\
  \hline 7 & Terrasoft & 657 654 & 476 561 & 38 \\
  \hline 8 & Телфин & 398 500 & 317 900 & 25 \\
  \hline 9 & МойСклад & 395 000 & 265 000 & 49 \\
  \hline 10 & ИТ-Град & 265 400 & 113 420 & 134 \\
  \hline
  \end{tabular}
\end{table}

\subsection{Угрозы облачной безопасности}

Своевременное определение основных угроз безопасности является первым шагом к минимизации рисков в сфере облачных вычислений.
Исследовательская группа Cloud Security Alliance, на конференции RSA в Сан-Франциско, выделила 12 основных угроз безопасности в сфере облачных вычислений за 2016~г. \cite{csa}:
\begin{itemize}
  \item утечка данных;
  \item компрометация учетных записей и обход аутентификации;
  \item взлом интерфейсов и API;
  \item уязвимость используемых систем;
  \item кража учетных записей;
  \item инсайдеры-злоумышленники;
  \item целевые кибератаки;
  \item перманентная потеря данных;
  \item недостаточная осведомленность;
  \item злоупотребление облачными сервисами;
  \item DDoS-атаки;
  \item совместные технологии, общие риски.
\end{itemize}

Как и в случае с традиционными инфраструктурами, облака подвергаются тем же угрозам.
Злоумышленники пытаются получить доступ к данным пользователей, которые хранятся в облаке.
Утечка данных компании может нанести значительный ущерб бизнесу, вплоть до банкротства.
Поставщики облачных услуг пытаются минимизировать риски утечки информации путем внедрения многофакторной аутентификации и стойкого шифрования.
К примеру все действия пользователя на портале самообслуживания должны осуществляться по безопасному протоколу HTTPS, а аутентификация в личный кабинет должна сопровождаться подтверждением через мобильный телефон или электронную почту.

Однако даже при использовании безопасного протокола, пользователь может использовать слабые пароли или управлять ключами шифрования ненадлежащим образом, например хранить их в публичном доступе.
Также компании могут сталкиваться с проблемами назначения прав пользователей, например при увольнении сотрудника или при переводе его на другую должность.
В таком случае, помимо повышения компетентности руководства компании, CSA рекомендует использовать одноразовые пароли, токены, USB-ключи, смарт-карты и прочие механизмы многофакторной аутентификации.
Использование таких механизмов значительно усложняет подбор паролей и компрометация инфраструктуры возможна только при наличии физического доступа, например к USB-ключу.

Использование пользовательского интерфейса и API значительно упрощает взаимодействие пользователя с услугой.
Однако появляется и дополнительный вектор для атаки злоумышленниками, так как информация предоставленная через API является строго конфиденциальной.
Особо важно в данном случае проработать механизмы контроля доступа и шифровать API.
Для предотвращения таких взломов необходимо периодически запускать тесты на проникновение, проводить аудит безопасности, моделировать угрозы и использовать прочие превентивные методы.

Наиболее популярная проблема при использовании облачных вычислений --- это уязвимости в используемых приложениях.
Зачастую, при использовании облачных IaaS-услуг, компании полностью полагаются на поставщика облачных услуг, хотя ответственность за размещаемые приложения и уязвимости в используемых системах лежит на пользователе.
Для предотвращения взлома приложения, необходимо регулярное сканирование на предмет наличия уязвимостей, мониторинг активности приложения, применение последних патчей безопасности.

Путем кражи учетных записей пользователей облачного окружения, злоумышленники внедряют фишинговые страницы и различные эксплоиты.
В таком случае стратегия <<защита в глубину>> может оказаться недостаточной.
CSA рекомендует в таком случае выполнять контроль учетных записей пользователей, вести журналы выполняемых транзакций, а также использовать механизмы многофакторной аутентификации.

Уволенные сотрудники могут преследовать различные цели, начиная от мести и заканчивая коммерческой выгодой от кражи данных, поэтому особо важно контролировать учетные записи всех сотрудников компании, ключи шифрования, доступы к внутренним сервисам.
Инсайдерские угрозы могут принести значительный ущерб компании, в случае утечки данных конкурентам.
Также важно проводить мониторинг, аудит и логирование действий учетных записей.

В настоящее время особую опасность представляют целевые кибератаки.
При таком методе взлома с помощью соответствующих инструментов можно добиться проникновения в облачную инфраструктуру, при этом обнаружить такое вторжение затруднительно.
Использование современных решений обеспечения безопасности позволяют вовремя обнаруживать такие кибератаки.
Необходим мониторинг подобных инцидентов и незамедлительная реакция на них, также стоит использовать профилактические меры в целях недопущения взлома.

Перманентная потеря данных не настолько страшна, если использовать резервное копирование.
Как правило, злоумышленники не ставят себе основной целью удаление данных, так как их можно восстановить при достаточной квалификации системных администраторов.
Использование ежедневного резервного копирования и периодическая проверка работоспособности этих данных позволяет избежать их полной потери.
Необходимо шифровать данные резервных копий, использовать альтернативные площадки для их размещения, разделять данные пользователей и данные приложений.

Компании, переходящие в облака часто не осведомлены о том, как это работает и с чем им придется в дальнейшем столкнуться.
Для решения данного вопроса необходимо понимать как работают облачные сервисы, какие риски берет на себя компания при заключении договора с поставщиком облачных услуг.
Персонал компании должен иметь соответствующую квалификацию для управления облачной услугой.

Облачные услуги могут использоваться злоумышленниками для совершения злонамеренных действий, таки как атак на отказ (\hyperlink{ddos}{DDoS}), рассылка спама, размещение фишинговх страниц и прочее.
Поставщики облачных услуг должны вести мониторинг таких пользователей, анализировать трафик и своевременно реагировать на подобные инциденты.

Особо актуальной угрозой на сегодняшний день стали DDoS-атаки.
Вычислительные мощности дешевеют, поэтому подобные атаки становятся все дешевле и сильнее для злоумышленника.
При выборе поставщика облачных услуг стоит обратить внимание, каким образом у них организована защита от подобных атак, так как при реальной угрозе провайдер может заблокировать пользователя и отказаться от предоставления услуг.

Облачные технологии базируются на большом количестве разнообразного программного обеспечения и протоколов.
Если в одном компоненте возникает уязвимость, то она напрямую влияет на всю инфраструктуру.
CSA рекомендует использовать стратегии <<безопасности в глубину>>, сегментации сети, использования многофакторной аутентификации, использования систем обнаружения вторжений.

\subsection{Тенденции развития облачных вычислений}

В настоящее время одними из основных тенденций развития в сфере ИТ, в облачных вычислениях в частности, являются <<зеленые>> центры обработки данных и программно-конфигурируемые сети или Software-defined Networking (\hyperlink{sdn}{SDN}).

Все центры обработки данных сталкиваются с проблемой вывода тепловой энергии.
Тепловая энергия являет побочным действием от эксплуатации плотно укомплектованных серверных стоек.
Для охлаждения ЦОД необходимо большое количество холодильных установок, которые потребляют количество энергии сравнимое с некоторыми городами \cite{cnewsdc}.
Крупные компании хотят строить центры обработки данных в северных странах, таких как Исландия и Финляндия, однако удаленность от крупных магистральных сетевых точек и недостаток квалифицированных кадров для обслуживания ЦОД замедляют развитие данной тенденции.
Тепло, выделяемое дата-центром можно использовать для обогрева близлежащих жилых комплексов.
Возможность заново использовать излишнее тепло от серверов предусматривается на этапе разработки нового ЦОД, помогая повысить энергетическую эффективность оборудования.

Одной из главных расходных статей в проектировании центров обработки данных таких крупных компаний как Google, Amazon, eBay, Microsoft, является электроэнергия.
В 2008~г. компания McKinsey \& Company провела аналитический обзор выбросов углекислого газа от выработки электроэнергии для нужд ЦОД \cite{greendc}.
Согласно исследованиям, суммарные показатели выброса углекислого газа ЦОД сравнимы с выбросам такой страны как Аргентина.

За счет использования энергии солнца и ветра уже работает большое количество коммерческих ЦОД.
Первый в мире <<зеленый>> дата-центр, который работает исключительно на энергии ветра построен в США, в штате Иллинойс.

Компания Microsoft использует контейнеры в качестве здания для ЦОД.
Таким образом такие контейнеры легко переносить к источникам возобновляемой энергии, за счет этого выбросы углекислого газа в атмосферу значительно сокращаются.

Также Microsoft в рамках Project Natick создала прототип ЦОД под названием Leona Philpot.
Прототип был погружен в США, недалеко от Тихого океана и успешно проработал на протяжении четырех месяцев.
Большое количество теплообменников, которыми был оснащен Leona Philpot, передавали лишнее тепло от серверов в воду.
При этом по оценкам экологов, это не влияет на глобальное повышение на температуру воды в мировом океане.

Бывший сотрудник компании Google Джон Данн предложил использовать уже не работающую электростанцию в качестве нового ЦОД.
Атомная электростанция Vermont Yankee использовалась с 1972~г. по 2014~г.
Так как АЭС находится близко к водным ресурсам и железной дороге, необходимо было лишь преобразовать здание и подвести к ЦОД сетевую инфраструктуру.

В России существует проект строительства ЦОД рядом с Калининской АЭС.
Таким образом корпорация Росэнергоатом обеспечит бесперебойным надежным источником электроэнергии для 4800 серверных стоек.

Основными тенденциями развития корпоративных сетей и сетей центров обработки данных являются:
\begin{itemize}
  \item рост объемов трафика, изменение структуры трафика;
  \item рост числа пользователей мобильных приложений и социальных сетей;
  \item высокопроизводительные кластеры для обработки большого количества данных;
  \item виртуализация для предоставления облачных услуг.
\end{itemize}

Программно-конфигурируемая сеть --- форма виртуализации вычислительных ресурсов (сети), в которой уровень управления отделен от уровня передачи данных (рис. \ref{vnet}).
Основным преимуществом такой сети является то, что большое количество сетей возможно программно объединить в одну и централизованно управлять ею.

\addimg{vnet}{1}{Схема программно-конфигурируемой сети}{vnet}

Если рассмотреть современный маршрутизатор или коммутатор, то логически его можно разделить на три компонента:
\begin{itemize}
  \item уровень управления;
  \item уровень управления трафиком;
  \item передача трафика.
\end{itemize}

На уровне управления присутствует командный интерфейс, программная оболочка, протоколы управления или встроенный веб-сервер.
Основная задача уровня управления --- обеспечить управление устройством.
На уровень управления трафиком приходятся алгоритмы.
Функциональная задача данного уровня --- автоматическая реакция на изменение трафика.
Функционал передачи трафика обеспечивает физическую передачу данных на низших уровнях.

Для построения программно-конфигурируемых сетей используется стандартный протокол OpenFlow.
Большое количество коммутаторов уже поддерживает этот протокол.
Крупные вендоры сетевого оборудования, такие как Hewlet-Packard, считают что развитие SDN и в частности протокола OpenFlow позволит возобновить процесс внедрения инноваций в уже давно устоявшейся области сетевых технологий.

\clearpage
