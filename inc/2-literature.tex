\section{Обзор литературных источников по тематике исследования} \label{literature}

Исторически, ситуация сложилась так, что слово <<облако>> используется в качестве метафоры сети Интернет.
Позже, оно было использовано для изображения Интернет в компьютерных сетевых диаграммах и схемах.

Облачные вычисления можно обозначить, как выделение ресурсов в облаке.
В соответствии с NIST (Национальный институт стандартов и технологий), формальное определение облачных вычислений заключается в следующем:
<<Облачные вычисления являются моделью обеспечения повсеместного, удобного доступа по требованию по сети, общему пулу конфигурируемых вычислительных ресурсов (например сетей, серверов, систем хранения данных (СХД), приложений и услуг), которые могут быстро и с минимальными усилиями предоставлены для управления поставщиком услуг>>. \cite{nist}

Согласно опросам института Понемона в 2016 году, среди 3476 респондентов в сфере информационной безопасности из Соединенных Штатов Америки, Великобритании, Австралии, Германии, Японии, Франции, Японии, России, Индии и Бразилии, 73\% респондентов так или иначе используют облачные вычисления в своей инфраструктуре.
Особый рост внедрения облачных сервисов произошел в последние 2 года. \cite{gemalto}

Хранение данных пользователей, почты и потребительских данных в облаке выросло в 2016 году по сравнению с 2014 годом.

\addimghere{cloud-data}{1}{Сравнение использования облачных вычислений для хранения данных}{cloud-data}

Провайдеры облачных услуг предлагают различные виды услуг, построенных поверх базового резервирования и освобождения ресурсов.
Большинство из этих услуг попадают в одну из следующих категорий:
\begin{itemize}
  \item инфраструктура как услуга (IaaS);
  \item платформа как услуга (PaaS);
  \item программное обеспечение как услуга (SaaS).
\end{itemize}

\addimghere{aas}{1}{Модели обслуживания облака}{aas}

Большинство провайдеров используют различные виды веб-интерфейса, на основе которого можно построить необходимый стек технологий.
Облачные провайдеры используют модель <<pay-as-you-go>>, в которой оплата производится только за время использования ресурсов.

Ключевыми функциями облачных вычислений являются:
\begin{itemize}
  \item скорость и масштабируемость, доступ к необходимым ресурсам можно получить одним щелчком мыши, что экономит время и обеспечивает гибкость, в зависимости от потребностей сервиса, можно легко масштабировать ресурсы как вверх, так и вниз;
  \item стоимость, снижение первоначальных затрат на развертывание инфраструктуры позволяет сосредоточиться на приложениях и бизнесе, облачные провайдеры имеют возможность заранее оценить стоимость, что значительно облегчает планирование бюджета;
  \item легкий доступ к ресурсам, пользователи могут получить доступ к инфраструктуре из любого места и устройства, до тех пор, пока существует подключение к провайдеру;
  \item обслуживание, все работы по техническому обслуживанию ресурсов осуществляются поставщиком облачных услуг, пользователи не должны беспокоиться об этом;
  \item мультиаренда, несколько пользователей могут использовать один и тот же пул доступных ресурсов;
  \item надежность, ресурсы могут быть размещены в разных дата-центрах, для обеспечения повышенной надежности.
\end{itemize}

Как правило, облако может быть развернуто согласно следующим моделям:
\begin{itemize}
  \item частное облако;
  \item публичное облако;
  \item гибридное облако.
\end{itemize}

\addimghere{pub-priv-hybr}{1}{Модели развертывания облака}{pub-priv-hybr}

Частное облако, эксплуатируется исключительно одной организацией, оно может быть размещено внутри или снаружи сети организации и управляться внутренними командами или третьей стороной.
Частное облако можно построить с использованием такого программного обеспечения, как OpenStack;

Публичное облако доступно для всех пользователей, любой может использовать его после предоставления данных кредитной карты.
AWS (Amazon Web Services) и GCE (Google Compute Engine) являются примерами публичных облаков;

Гибридное облако, является результатом объединения публичного и частных облаков.
Гибридное облако может быть использовано для хранения секретной информации о частном облаке, предлагая при этом услуги на основе этой информации из публичного облака.

В вычислениях, виртуализация является процессом создания виртуальной (не физической) версии чего-либо, в том числе аппаратных платформ виртуального компьютера, операционных систем, устройств хранения данных и вычислительных ресурсов.

Виртуализация может быть предоставлена на различных аппаратных и программных уровнях, таких как центральный процессор, диск, память, файловые системы и прочее.
Чаще всего виртуализация используется для создания виртуальных машин и эмуляции различного оборудования для последующей установки операционных систем (ОС) на них.

Виртуальные машины создаются на основе гипервизора, который работает поверх операционной системы хост-компьютера (физического компьютера, не виртуального).
С помощью гипервизора возможна эмуляция аппаратных средств, таких как процессор, диск, сеть, память, а также установка гостевых операционных систем на них.
Возможно создание нескольких гостевых виртуальных машин с различными операционными системами на гипервизоре.
Например, можно взять машину на Linux и установить ее на <<голое>> железо (bare-metal), и после настройки гипервизора возможно создание нескольких гостевых машин на Linux и Windows.

На данный момент все современные процессоры поддерживают аппаратную виртуализацию, это необходимо для безопасного и эффективного обмена ресурсами между хост-системой и гостевыми системами.
Большинство современных процессоров и гипервизоров также поддерживают вложенную виртуализацию, что позволяет создавать виртуальные машины внутри виртуальных машин.




---

Актуальность: облака везде, облака нужны всем, не только бизнес-клиентам, но и обычным людям.

Т.к. популярность облаков появилась сравнительно недавно и она стремительно развивается, не всегда успевают учесть все аспекты безопасности.

Также из-за того, что облако состоит из большого количества ПО на различных уровнях, нужно учитывать все уязвимости, так как они могут всплыть на каждом из уровней.

Мысли:

* клиенты слабо представляют насколько защищены облака, поэтому предпочитают частное облаков, взамен публичного, не доверяют провайдеру

* основные аспекты облаков: мониторинг, управление, безопасность, доступность

* если надо добавить часть по экономике - файл 124.pdf

* по поводу иаас, преимущества понятны, а вот минусы в том, что если получают доступ к хост-ноде, то все, также это может быть изнутри, например уязвимость на гипервизоре

* конкретные проблемы с табличками описаны тут 1608.08787v1.pdf

* файл cloud-security-study-report.pdf конкретный отчет сравнение использования облаков в 2016 году по сравнению с 2014

* в файле informatsionnaya-bezopasnost-pri-oblachnyh-vychisleniyah-problemy-i-perspektivy.pdf хорошо расписан вопрос по стандартизации облаков, также кратко написано про риски использования облаков

* тут тоже про стандарты psta2011-4-17-31.pdf

* в файле str50.pdf рассказывакется про проблемы в России, проблемы с точки зрения инф. безопасности

* реальные опросы от интела 2012г, которые рассказывают, что препятствуют уходу в облака, файл whats-holding-back-the-cloud-peer-research-report.pdf

* хорошая презентация Zegzhda-PD-supernova-2.pdf краткие тезисы, примеры картинок, модель безопасности даже есть

---



\addimghere{gartner-providers}{1}{Тестовая картиночка}{gartner-providers}
\addimghere{gartner-virt}{1}{Тестовая картиночка}{gartner-virt}

* что такое облака

* сравнение облаков 2014/16 тут графики

* технологии облачных вычислений soa/asp/virt... тут картинки

* saas/paas/iaas тут картинка

* гибрдное/публ/прив облако тут картинка

* стандартизация nist... тут можно табличку

* облачные провайдеры тут табличка/картинка

* специфика облаков в России

* тенденции развития облаков в мире (зеленые цоды, уход в виртуализированные хранилища и сети) пример картинки какой-то

* кратко по безопасности, основные пункты, кем регламентируется, как обеспечивается

Тут 10-20 страниц, без подпунктов, сплошной текст с картиночками, в общем теория.

* конкретный упор на безопасность уже в описании работы

\clearpage
