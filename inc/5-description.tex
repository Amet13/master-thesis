\section{Описание системы информационной безопасности облачной среды}

В разделе описания облачной инфраструктуры представлена структурная схема облачной инфраструктуры, а также структурная схема системы безопасности.

\subsection{Структурная схема облачной инфраструктуры}

Структурная схема облачной инфраструктуры представлена на рис. \ref{infrast-scheme}.

\addimghere{infrast-scheme}{1}{Структурная схема облачной инфраструктуры}{infrast-scheme}

В ЦОД 1 располагается основная часть инфраструктуры: сервера виртуального хостинга, виртуализации OpenVZ и KVM, сервер резервного копирования и выделенные сервера клиентов.

В ЦОД 2 на двух арендованных виртуальных машинах находится один из подчиненных DNS-серверов, а также сервер мониторинга.
На физических серверах располагаются важные элементы инфраструктуры: DNS-сервера, система биллинга и система управления IP-адресами.

\subsection{Структурная схема системы безопасности}

Структурная схема системы безопасности представлена на рис. \ref{cwpp}.

\addimghere{cwpp}{1}{Структурная схема системы безопасности облачной инфраструктуры}{cwpp}

Среди них можно выделить несколько наиболее важных.
Наиболее важными компонентами схемы являются:
\begin{itemize}
  \item контроль физического доступа;
  \item регулярное обновление программного обеспечения;
  \item централизованный мониторинг;
  \item проведение тестов на поиск уязвимостей.
\end{itemize}

\clearpage