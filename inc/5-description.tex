\section{Безопасность облачных вычислений}

Крупные поставщики облачных услуг в своих отчетах сообщали о печальных инцидентах в 2008~г. и 2009~г. \cite{amazon}.
Сервис компании Amazon под названием Simple Storage, а также Google Docs подверглись утечке информации пользователей.
Сервис Gmail, в феврале 2009~г. был недоступен более двух часов \cite{gmail}.
Серьезные уязвимости были обнаружены в программном обеспечении VMware для Mac в мае 2009~г.
Облачная платформа Azure от компании Microsoft была недоступна 22 часа в марте 2009~г. \cite{azure}.
Подобные серьезные инциденты безопасности могли привести к полному краху облачных услуг поставщиков.
Поставщик облачных СХД LinkUp допустил потерю 45\% данных пользователей и вынужден был закрыться в августе 2008~г. \cite{linkup}.

Меры по контролю безопасности в облаке аналогичны тем, которые существуют в традиционной ИТ-среде.
Однако облачные вычисления могут столкнуться с различными рисками и вызовами, из-за многопользовательских характеристик, моделей предоставления услуг и моделей развертывания.

\subsection{Проблемы безопасности облачных вычислений}

Безопасность облачных вычислений является частью развития компьютерной и сетевой безопасности, а также, в более широком плане информационной безопасности.

\subsubsection{Безопасность облачных вычислений}

Безопасность относится к широкому набору стратегий, технологий и средств управления, использующихся для защиты данных и приложений, а также связанной с ней инфраструктуры облачных вычислений.
Она представляет собой ответ на знакомый набор вызовов безопасности, проявляющихся в облаке.
Безопасность содержит набор политик, технологий и средств управления, предназначенных для защиты данных, инфраструктуры и клиентов от нападения, она позволяет соответствовать нормативным требованиям.
Большая безопасность обеспечивается при использовании многоуровневого подхода, когда каждый уровень интегрирован в общую систему управления, таким образом обеспечивается защита независимо от модели доставки.

\subsubsection{Вопросы безопасности в соответствии с SPI}

Безопасности инфраструктуры в соответствии с индексом параметра обеспечения безопасности (\hyperlink{spi}{SPI}) соответствуют безопасность уровня сети, уровня хоста и уровня приложений.
Инфраструктура безопасности в большей степени применима для IaaS-клиентов, но также может быть применена для PaaS и SaaS, поскольку они имеют последствия для риска и соответствия руководству клиентов.

На сетевом уровне важно проводить различие между публичными и частными облаками.
В частных облаках нет никаких новых атак, уязвимостей или изменений риска, которые должны быть приняты во внимание.
С другой стороны, в публичных облаках, сети клиентов и облачного провайдера должны взаимодействовать между собой, что вносит изменения в требования безопасности.

Существует несколько факторов риска в этом:
\begin{itemize}
  \item обеспечение конфиденциальности и целостности данных;
  \item обеспечение надлежащего контроля доступа (аутентификации, авторизации и аудита);
  \item обеспечение доступности интернет-ресурсов;
  \item замена установленной модели сетевых зон и уровней с доменами.
\end{itemize}

Данные, поступающие от поставщика публичного облака проходят через Интернет и это необходимо учитывать при обеспечении их конфиденциальности и целостности.
Примером является Amazon Web Services, где в декабре 2008~г. использовался протокол HTTP вместо HTTPS, таким образом увеличивая риск изменения данных без ведома пользователя \cite{aws}.

Отсутствие операций аудита сети облачного провайдера уменьшает доступ клиента к соответствующим журналам сети и данным.
Этот фактор риска может привести к повторному использованию IP-адресов, которые находятся в кэше \hyperlink{dns}{DNS} и могут ввести в заблуждение пользователей.

Еще одним из факторов риска является \hyperlink{bgp}{BGP}-перехват (префиксный перехват или перехват маршрута) является незаконным поглощением групп IP-адресов, повреждая интернет-таблицы маршрутизации.
Из-за ошибки конфигурации, можно присвоить себе автономную систему (\hyperlink{as}{AS}) без разрешения владельца.
Возможно и преднамеренное присвоение AS, но это случается гораздо реже.

Перехват префикса не является чем-то новым, однако из-за более широкого использования облачных вычислений, доступность облачных ресурсов увеличивается, а значит увеличивается количество мишеней для атак.
Атаки на систему доменных имен (DNS) также не являются чем-то новым, однако и она представляет угрозу в сети из-за увеличения количества внешних DNS-запросов.
В дополнение к уязвимости в протоколе DNS и его реализации, кэш DNS может подвергаться атакам.
DNS-север может принимать некорректную информацию таким образом, что имя сервера целевого домена перенаправляется на другой целевой домен.
DDoS также активно используется в атаках на IaaS.
В таких протоколах как DNS, \hyperlink{ntp}{NTP}, \hyperlink{snmp}{SNMP} используется атака, при которой на маленький запрос к серверу генерируется огромный ответ, таким образом даже небольшим количеством ресурсов можно добиться, что весь сетевой канал будет занят паразитным трафиком.
Кроме внешних атак, внутренний DDoS-атаки могут быть осуществлены через сеть поставщика IaaS-услуг.
Скорее всего провайдер не контролирует подобное, поэтому предотвращение таких инцидентов может занять много времени.
Только клиенты могут предотвратить атаки такого характера.

Традиционная сетевая безопасность зависит от изоляции модели сетевых зон и уровней, где пользователи и системы имеют доступ только к определенным зонам.
Тем не менее, эта модель не может быть применена к публичным облакам IaaS и PaaS, но этот подход можно заменить в публичных облаках так называемыми <<группами безопасности>>, <<доменами безопасности>> или <<виртуальными центрами обработки данных>>.
Эти группы безопасности могут позволить виртуальным машинам иметь доступ друг к другу с помощью виртуального межсетевого экрана, фильтрации трафика на основе IP-адреса, типов пакетов и портов.
Кроме того, приложения логически сгруппированы на основе имен доменов.

Вопросы безопасности промежуточного узла (хоста) тесно связаны с различными моделями предоставления облачных услуг и моделей развертывания.
Хотя нет никаких новых угроз для хостов, которые являются специфическими для облачных вычислений, некоторые угрозы безопасности виртуализации (побег из виртуальной машины, слабый контроль над гипервизором, ошибки в конфигурации системы) сопровождаются последствиями.
Эластичность облака может принести новые оперативные задачи с точки зрения обеспечения безопасности, которые могут быть гораздо сложнее обычных.

Так как злоумышленники могут использовать знания о конфигурации облака для вторжения в облачные сервисы, провайдеры облачных услуг публично не делятся информацией о используемых платформах, операционных системах и процессах.
Таким образом, безопасность хоста не может быть детально известна пользователям и вся ответственность за обеспечение безопасности лежит на поставщике облачных услуг.
Клиенты могут попросить провайдера обмениваться информацией в рамках соглашения о неразглашении (\hyperlink{nda}{NDA}), либо с помощью механизма оценки управления --- ISO/IEC 27002 или SysTrust, чтобы дать клиенту гарантию.
Для обеспечения виртуализации, поставщик облачных услуг использует гипервизоры Xen и VMware, которые являются bare-metal гипервизорами, то есть включены в состав существующей операционной системы.
Таким образом уровень абстракции предоставляется пользователю, при этом скрывая операционную систему хоста от конечных пользователей.
В случае SaaS, этот слой доступен только для разработчиков и персонала провайдера, а в случае PaaS, клиенты используют PaaS API, который взаимодействует с принимающим уровнем абстракции.
Хотя поставщик облачных услуг несет ответственность за безопасность хоста для SaaS и PaaS услуг, клиент может управлять рисками хранения информации, размещенной в облаке.

При использовании IaaS, клиенты сами несут ответственность за безопасность, которая может быть отнесена к категории безопасности программного обеспечения и безопасности виртуального сервера.

Управление программным обеспечением для виртуализации относится к провайдеру, в то время как клиенты не имеют доступа к этому программному обеспечению, особенно если это публичное облако.
Виртуализация аппаратного обеспечения или операционной системы позволяет совместно использовать ресурсы на нескольких гостевых виртуальных машинах одновременно и не мешая друг другу.
Виртуализации уровня хоста может быть осуществлено с использованием гипервизоров первого уровня, таких как VMware ESX, Xen, Oracle VM, Hyper-V.

Клиенты имеют полный доступ к виртуальному экземпляру операционной системы, который виден из Интернета и изолирован от других экземпляров технологиями гипервизора.
В таком случае безопасность и управление безопасностью лежит на пользователе.
Публичные IaaS могут быть очень уязвимыми и стать жертвами новых направлений мошенничества, таких как кража закрытых \hyperlink{ssh}{SSH}-ключей для доступа к хосту, кража счетов, атака на брандмауэры, развертывание троянов в виртуальных машинах.
Для обеспечения высокого уровня безопасности виртуального сервер должны использоваться сильные операционные процедуры в сочетании с автоматизацией процедур:
\begin{itemize}
  \item использование безопасной по умолчанию конфигурации --- построение пользовательских образов виртуальных машин, которые имеют только возможности и услуги, необходимые для поддержки стека приложений;
  \item отслеживание обновлений гостевых операционных систем;
  \item защита целостности шаблонов от несанкционированного доступа;
  \item хранение закрытых ключей в безопасном месте;
  \item для доступа к командной оболочке не использовать пароли, а только SSH-ключи;
  \item периодически просматривать журналы для анализа подозрительных действий.
\end{itemize}

Безопасность приложений является одним из важнейших элементов безопасности инфраструктуры.
Проектирование и внедрение приложений для облачной платформы требуют совершенствования существующей практики и стандартов для существующих программ безопасности приложений.
Приложения варьируются от автономных однопользовательских приложений до сложных приложений электронной коммерции, многопользовательских, используемых миллионами пользователей, но наиболее уязвимыми являются веб-приложения.
Так как браузер появился в качестве клиента конечного пользователя для доступа к облачным приложениям, важно, чтобы безопасность браузера была включена в сферу безопасности приложений.
Таким образом, пользователям рекомендуется регулярно проверять обновления браузера для поддержания безопасности.

Согласно некоторым исследованиям, почти половина уязвимостей относится к веб-приложениям \cite{cis}.
Рейтинг OWASP Top 10 показывает, что основными угрозами для безопасности приложений являются \hyperlink{sql}{SQL}-инъекции, межсайтовый скриптинг (\hyperlink{xs}{XSS}), взломанные сессии, небезопасные ссылки, злонамеренное исполнение файлов и другие уязвимости, которые являются результатом ошибок программирования и конструктивных недостатков \cite{owasp}.

Веб-приложения, созданные и развернутые на платформе открытого облака легко сканировать, обладая соответствующими знаниями и инструментами злоумышленника, поэтому эти приложения должны быть разработаны в соответствии с моделью безопасности.
Приложения должны включать в себя полный цикл разработки программного обеспечения, с проектированием, кодированием, тестированием и выпуском.

Ответственность за безопасность веб-приложений в облаке лежит как на поставщике облачных услуг, так и на пользователе и зависят от модели доставки облачных сервисов и соглашении об уровне обслуживания.
В модели SaaS, поставщик услуг в основном отвечает за свою собственную безопасность приложений, в то время как клиент отвечает за оперативные функции безопасности и управления пользователями и доступом.
Интересная проблема заключается в функции аутентификации и контроле доступа, предлагаемыми поставщиком SaaS.
Различные поставщики предлагают разные методы: веб-инструменты администрирования пользовательского интерфейса для управления аутентификацией и управления доступом приложения (Salesforce, Google), встроенные функции, которые пользователи могут вызывать, назначать привилегии чтения и записи другим пользователям (Google Apps).
Пользователи должны принять эти механизмы контроля доступа, а также включать в себя управление привилегиями на основе ролей и функций пользователя и реализовать политику надежного пароля.

Так как PaaS-облака поддерживают не только среду, но и собственные приложения пользователя, безопасность приложений можно разделить на два уровня: безопасность платформы PaaS и безопасность клиентских приложений, развернутых на PaaS.
Поставщики облачных услуг несут ответственность за обеспечение работы стека программного обеспечения или гарантируют безопасность приложений сторонних разработчиков.
Они также отвечают за мониторинг новых ошибок и уязвимостей и мониторинг совместно-используемой сетевой и системной инфраструктуры приложений.
Разработчики PaaS должны разбираться в API конкретных облачных провайдеров и его функции безопасности: объекты безопасности и веб-сервисы для настройки аутентификации и авторизации инструментов в приложениях.

В IaaS, пользователи имеют полный контроль над своими приложениями, потому что весь стек ПО работает на виртуальных серверах клиента, и от провайдера облачных услуг они должны получать только основные рекомендации со ссылкой на политику брандмауэра.
Веб-приложения, развернутые в публичном IaaS-облаке должны быть разработаны для модели Интернет, и должны периодически проверяться на наличие уязвимостей.
Разработчики IaaS должны реализовать свои собственные функции для обработки аутентификации и авторизации и должны сделать их применение благоприятной для сервисных функций аутентификации.

\subsection{Проблемы безопасности облачных вычислений}

Идея защиты данных в облаке аналогична традиционной безопасности данных, но из-за открытости и мультитенантности, безопасность данных в облаке имеет свою специфику.

\subsubsection{Жизненный цикл данных}

Жизненный цикл данных относится ко всему процессу от его создания и до разрушения.
Жизненный цикл данных делится на шесть этапов (рис. \ref{lifecycle}) \cite{ls}.

\addimg{lifecycle}{0.8}{ Жизненный цикл безопасности данных}{lifecycle}

Создание является генерацией нового или модификацией существующего элемента цифровых данных, поэтому этот цикл может быт назван <<создание или обновление данных>>.
Это может быть любой вид контента, а не просто документ или база данных, и может быть структурированным или неструктурированным.
На этом этапе информация является секретной.

Хранение является актом представления цифровых данных в структурированном или неструктурированном виде (базы данных или файлы), как правило, процесс происходит одновременно с созданием.
Здесь классификация и права на управления безопасностью отображаются, включая контроль доступа, шифрование и управление правами.

В цикле использования данные рассматриваются, обрабатываются или иным образом используются в какой-либо деятельности, не включая модификации.
Эти средства управления применяются к данным в момент использования, как правило, пользователя или приложения.
Существуют средства управления, такие как мониторинг деятельности, превентивное управление, управление правами и логические элементы управления, которые обычно применяются в базах данных и приложениях.

В процессе распространения данные становятся доступными для других и обмениваются между пользователями, клиентами и партнерами.
Эти элементы управления включают в себя сочетание обнаружительного и превентивного контроля, шифрование для безопасного обмена данными, а также логические средства управления и безопасности приложений.

На этапе архивации данные мало используются и входят долгосрочное хранение.
Здесь защита данных и их доступность обеспечивается с помощью комбинации шифрования и управления активами.

Данные могут быть уничтожены с помощью физических или цифровых средств.
Данные должны быть удалены бесследно, необходимо использовать инструменты отслеживая любых копий.

\subsubsection{Аспекты безопасности данных}

Обеспечение безопасности данных в облаке требует различных подходов и хорошего понимания на различных этапах передачи данных:
\begin{itemize}
  \item данные в пути;
  \item данные в состоянии покоя;
  \item обработка данных;
  \item родословная данных;
  \item происхождение данных;
  \item остаточные данные.
\end{itemize}

Первые три этапа можно сравнить с тривиальным примером использования электронной почты, где отправка электронного сообщения представляет данные в процессе их перевозки, данные в состоянии покоя электронной почты в почтовом ящике и обработка данных относится к печатаемому ответу.

Рекомендуется защищать данные с помощью передачи с использованием алгоритмов шифрования имеющих необходимые допуски, особенно при использовании публичного облака.
Использование зашифрованных данных по незащищенному протоколу может обеспечить конфиденциальность, но не обеспечивает целостность данных.
Следовательно, необходимо использовать протокол, который будет обеспечивать целостность и конфиденциальность данных, например, \hyperlink{ftp}{FTP} через \hyperlink{ssl}{SSL} (FTPS), безопасный протокол передачи гипертекста (\hyperlink{http}{HTTP/2}) или Secure Copy Program (\hyperlink{scp}{SCP}).

Шифровать данные в состоянии покоя не так просто, как данные в пути.
Шифрование возможно при использовании облачного IaaS-сервиса для хранения данных, но использование шифрования в PaaS и SaaS не всегда осуществимо, поскольку он препятствует индексации и поиску этих данных.
Однако, когда приложение работает с обработкой данных, эти данные должны быть в незашифрованном виде, и это сценарий, при котором данные организации, не шифруются.
На данный момент полностью гомоморфная схема шифрования все еще разрабатывается \cite{ibm}, однако это сложно осуществить, так как требуются значительные вычислительные мощности.

Более сложной проблемой для клиентов является предоставление данных, что означает не только доказательство целостности данных, но более конкретное происхождение данных.
Происхождение данных представляет собой вид метаданных, содержащий историю данных, начиная от первоначальных источников в хранилище данных.

Последний аспект безопасности данных является остаточная информация --- остаточное представление цифровых данных, которое сохраняется даже после попытки удалить или стереть данные.
Некоторые операционные системы не удаляют данные немедленно, когда пользователь делает запрос на удаление, система перемещает его в зону ожидания, откуда данные можно для легко восстановить от возможного сбоя или ошибки.
На обычных жестких дисках есть остаточная намагниченность, которую можно устранить большим количеством циклов перезаписи.

Облачные вычисления с ее использованием виртуализации усложняет остаточную намагниченность данных.
Перезапись физических носителей практически невозможна, так как облачная инфраструктура может распределять данные клиента или экземпляр виртуальной машины на несколько физических дисков.
Кроме того, данные, записанные на отдельных дисках остаются до тех пор, пока провайдер перераспределяет сектора.
Для задачи получения данных из облака, первая рекомендация состоит в том, чтобы зашифровать данные, прежде чем они будут храниться в облаке.
Получение ключей, рекомендуется осуществлять локально.
Используя эти рекомендации, данные могут быть надежно удалены путем простого удаления ключа.
Такой подход является достаточным, если облако используется только для хранения зашифрованных данных, а не для его обработки, так как данные должны быть расшифрованы \cite{zap}.

\subsection{Обеспечение безопасности инфраструктуры Google}

В январе 2017~г. компания Google опубликовала документ под названием <<Google Infrastructure Security Design Overview>>, в котором она описывает ряд мер, которые предпринимаются компанией для обеспечения безопасности серверной инфраструктуры \cite{google}.

Некоторые особенности:
\begin{itemize}
  \item все серверные компоненты, используемые в инфраструктуре, разработаны инженерами Google;
  \item программные компоненты систем проверяются по цифровой подписи;
  \item при размещении серверного оборудования в чужих ЦОД, Google использует собственный периметр физической безопасности;
  \item данные записываются на носители только в зашифрованном виде, при этом шифрование реализуется на аппаратном уровне;
  \item жизненный цикл каждого накопителя жестко контролируется;
  \item для обеспечения виртуализации используется модифицированный гипервизор KVM;
  \item отдельные сервисы в виртуальных машинах изолируются на уровне контейнеров;
  \item весь программный код компании проходит ряд тщательных проверок, вплоть до ручного рецензирования экспертами по безопасности и криптографии;
  \item исходные коды хранятся в одном централизованном репозитории, бинарные файлы используемые в инфраструктуре, могут быть собраны только из определенной ревизии;
  \item внутренняя сетевая безопасность основана на разделе полномочий и уровнях доступа к сервисам;
  \item обмен данными между сервисами осуществляется только в зашифрованном виде с использованием криптографических идентификаторов;
  \item персонал использует только двухфакторную аутентификацию и имеет доступ только к разрешенным сервисам;
  \item действия сотрудников, требующие расширенных привилегий по возможности автоматизированы.
\end{itemize}

\clearpage
