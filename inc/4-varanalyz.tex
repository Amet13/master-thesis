\section{Вариантный анализ}

Выберем гипервизор для организации виртуализации при помощи метода анализа иерархии (\hyperlink{mai}{МАИ}).
Метод состоит в разложении проблемы на все более простые составные части и дальнейшей обработке последовательных суждений лица принимающего решение по парным сравнениям.
В результате может быть выражена интенсивность или относительная степень взаимодействия элементов в иерархии.
В результате получаются численные выражения этих суждений.
МАИ включает в себя процедуры синтеза множественных суждений, получение приоритетных критериев и нахождение альтернативных решений.
Полученные знания являются оценками в шкале отношений и соответствуют жестким оценкам \cite{var-analyz}.

В качестве альтернатив используются различные гипервизоры, на которых может быть реализована виртуализация для облачной среды.
В зависимости от выбранного гипервизора будет выбран тот или другой интерфейс взаимодействия с пользователем, так как каждая система имеет в своем составе такой интерфейс.
Критерием эффективности являются: цена, масштабируемость, отказоустойчивость, интерфейсы управления.

Альтернативы, которые участвуют в вариантном анализе:
\begin{itemize}
  \item \hyperlink{kvm}{KVM} (альтернатива А);
  \item Hyper-V (альтернатива Б);
  \item VMware vSphere (альтернатива В).
\end{itemize}

Критерии, по которым выбирается тот, или иной гипервизор:
\begin{itemize}
  \item цена (А1);
  \item масштабируемость (А2);
  \item отказоустойчивость (А3);
  \item интерфейсы управления (А4).
\end{itemize}

\subsection{Построение матриц парных сравнений второго уровня}

На основе вышеперечисленных критериев построим матрицу парных сравнений второго уровня, где строки и столбцы составляют выбранные критерии.
Сравнение критериев проведём по шкале относительной важности согласно с табл. \ref{crit}.
\begin{table}[H]
  \caption{Оценка критериев}\label{crit}
  \begin{tabular}{|p{4cm}|p{12cm}|}
  \hline Интенсивность относительной важности & \multicolumn{1}{|c|}{Определение} \\
  \hline 1 & если элементы $A_i$ и $A_k$ одинаково важны \\
  \hline 3 & если элементы $A_i$ и $A_k$ одинаково важны \\
  \hline 5 & если элемент $A_i$ значительно важнее элемента $A_k$ \\
  \hline 7 & если элемент $A_i$ явно важнее элемента $A_k$ \\
  \hline 9 & если элемент $A_i$ по своей значимости абсолютно превосходит элемент $A_k$ \\
  \hline 2,4,6,8 & используются для облегчения компромиссов между оценками, слегка отличающимися от основных чисел \\
  \hline
  \end{tabular}
\end{table}

В результате выполнения попарных сравнений, построили матрицу, представленную в табл. \ref{matrix}.
\begin{table}[H]
  \caption{Матрица попарных сравнений второго уровня}\label{matrix}
  \begin{tabular}{|l|l|l|l|l|l|}
  \hline \multicolumn{2}{|c|}{Критерии} & A1 & A2 & A3 & A4 \\
  \hline A1 & Цена & 1 & 1/5 & 1/7 & 3 \\
  \hline A2 & Масштабируемость & 5 & 1 & 1/5 & 7 \\
  \hline A3 & Отказоустойчивость & 7 & 5 & 1 & 8 \\
  \hline A4 & Интерфейсы управления & 1/3 & 1/7 & 1/8 & 1 \\
  \hline
  \end{tabular}
\end{table}

\subsection{Вычисление вектора приоритетов для матрицы парных сравнений второго уровня}

Из группы матриц попарных сравнений формируется набор локальных приоритетов, которые выражают относительное влияние множества элементов на элемент примыкающего сверху уровня.
Сначала вычислим геометрическое среднее в каждой строке матрицы A по формуле (\ref{geomean}):
\begin{equation}\label{geomean}
b_i = \sqrt[n]{\prod_{k=1}^{n}a_{ik}}
\end{equation}

Проведем вычисления компонент вектора локальных приоритетов:

$b_1 = \sqrt[4]{1 \cdot 0,2 \cdot 0,1429 \cdot 3} = \sqrt[4]{0,0857} = 0,5411$

$b_2 = \sqrt[4]{5 \cdot 1 \cdot 0,2 \cdot 7} = \sqrt[4]{7} = 1,6266$

$b_3 = \sqrt[4]{7 \cdot 5 \cdot 1 \cdot 8} = \sqrt[4]{280} = 4,0906$

$b_4 = \sqrt[4]{0,3333 \cdot 0,1429 \cdot 0,125 \cdot 1} = \sqrt[4]{0,006} = 0,2783$

Просуммируем полученные значения:

$B = 0,5411 + 1,6266 + 4,0906 + 0,2783 = 6,5355$

Определим значения компонент вектора локальных приоритетов по формуле (\ref{veccom}):
\begin{equation}\label{veccom}
x_i = \frac{b_i}{B}, i = \overline{1,n}
\end{equation}

Выполним расчеты:

$x_1 = \frac{b_1}{B} =\frac{0,5411}{6,5355} = 0,0828$

$x_2 = \frac{b_2}{B} =\frac{1,6266}{6,5355} = 0,2489$

$x_3 = \frac{b_3}{B} =\frac{4,0906}{6,5355} = 0,6259$

$x_4 = \frac{b_4}{B} =\frac{0,2783}{6,5355} = 0,0426$

Так как числа $b_i$ нормализуются делением каждого числа на сумму всех чисел, то должно выполняться условие 
(\ref{vecsum}):
\begin{equation}\label{vecsum}
\sum_{i=1}^{n} x_i = 1, i = \overline{1,n}
\end{equation}

В итоге:

$X = 0,0828 + 0,2489 + 0,6259 + 0,0426 = 1,0002$

Погрешность в $0,0002$ допустима и является следствием округления до четвертого знака.

\subsection{Исследование на согласованность матрицы парных сравнений второго уровня}

Оценим отношение согласованности для матрицы попарных сравнений второго уровня по формуле (\ref{atcon}):
\begin{equation}\label{atcon}
y_i = \sum_{i=1}^{n} a_{ik}, k = 1,2,...,n
\end{equation}

Выполним расчеты:

$y_1 = 1 + 5 + 7 + 0,3333 = 13,3333$

$y_2 = 0,2 + 1 + 5 + 0,1429 = 6,3429$

$y_3 = 0,1429 + 0,2 + 1 + 0,125 = 1,4679$

$y_4 = 3 + 7 + 8 + 1 = 19$

Вычислим наибольшее собственное значение матрицы сравнений согласно формуле (\ref{maxm}):
\begin{equation}\label{maxm}
\lambda_{max} = \sum_{i=1}^{n} x_i \cdot y_i
\end{equation}
где $x_i$ --- значения компонент вектора локальных приоритетов.

$\lambda_{max} = 0,0828 \cdot 13,3333 + 0,2489 \cdot 6,3429 + 0,6259 \cdot 1,4679 + 0,0426 \cdot 19 = 1,104 + 1,5787 + 0,9188 + 0,8094  = 4,4109$

Положительная обратно-симметричная матрица является согласованной тогда и только тогда, когда порядок матрицы и ее наибольшее собственное значение совпадают ($\lambda_{max} = n$).

Если элементы положительной обратносимметричной согласованной матрицы A изменить незначительно, то максимальное собственное значение $\lambda_{max}$ также изменится незначительно.
Если $\lambda_{max} \neq n$, всегда $\lambda_{max} > n$.

Как и ожидалось:

$\lambda_{max} = 4,4109 > n = 4$

В качестве степени отклонения положительной обратно-симметричной матрицы A от согласованной матрицы принимается отношение (\ref{icon}):
\begin{equation}\label{icon}
\text{ИС} = \frac{\lambda_{max} - n}{n - 1}
\end{equation}
которое называется индексом согласованности (\hyperlink{is}{ИС}) матрицы А и является показателем близости этой матрицы к согласованной.

Вычислим индекс согласованности для данной задачи:

$\text{ИС} = \frac{4,4109 - 4}{3} = 0,137$

Теперь необходимо сравнить значение индекса согласованности со значением случайной согласованности (\hyperlink{ss}{СС}).
\begin{table}[H]
  \caption{Случайная согласованность}\label{randcon}
  \begin{tabular}{|p{4cm}|l|l|l|l|l|l|l|l|l|l|}
  \hline Размер матрицы n & 1 & 2 & 3 & 4 & 5 & 6 & 7 & 8 & 9 & 10 \\
  \hline Случайная согласованность & 0 & 0 & 0,58 & 0,9 & 1,12 & 1,24 & 1,32 & 1,41 & 1,45 & 1,49 \\
  \hline
  \end{tabular}
\end{table}

Если разделить индекс согласованности на число, соответствующее случайной согласованности матрицы того же порядка, получается отношение согласованности (\hyperlink{ots}{ОС}), формула (\ref{acon}):
\begin{equation}\label{acon}
\text{ОС} = \frac{\text{ИС}}{\text{СС}} \cdot 100\%
\end{equation}

Величина отношения согласованности должна быть порядка 10\% или менее, чтобы быть приемлемой.
На практике же допускается значение, не превышающее 20\% \cite{mai}.
Если значение отношения согласованности выходит из этих пределов, то экспертам нужно исследовать задачу и пересмотреть суждения:

$\text{ОС} = \frac{0,137}{0,9} \cdot 100\% = 15,2\%$

Полученное значение $\text{ОС} = 15,2\% < 20\%$.
Считаем, что матрица попарных сравнений второго уровня является согласованной.

Оценки предпочтений критериев лица принимающего решение (\hyperlink{lpr}{ЛПР}) представлены в табл. \ref{marks}.
\begin{table}[H]
  \caption{Численные оценки предпочтений критериев ЛПР}\label{marks}
  \begin{tabular}{|l|l|l|l|}
  \hline \multicolumn{2}{|c|}{Критерии} & Место & Вес \\
  \hline A3 & Отказоустойчивость & 1 & 0,6259 \\
  \hline A2 & Масштабируемость & 2 & 0,2489 \\
  \hline A1 & Цена & 3 & 0,0828 \\
  \hline A4 & Интерфейсы управления & 4 & 0,0426 \\
  \hline
  \end{tabular}
\end{table}

Исходя из вычисленных значений численных оценок предпочтения делаем вывод, что критерий <<Отказоустойчивость>> является наиболее важным.
Критерий <<Масштабирование>> имеет существенный вес, а критерии <<Цена>> и <<Интерфейсы управления>> малозначимы.

\subsection{Построение матриц попарных сравнений третьего уровня}

На третьем уровне МАИ для каждого критерия проводятся попарные сравнения альтернатив и реализуются этап синтеза локальных приоритетов $z_j$ (j --- номер альтернативы, $j = \overline{1,m}$ в нашем примере $m = 3$) в соответствии с
формулами (\ref{geomean}) ... (\ref{acon}).
Также проводится исследование матрицы на согласованность \cite{var-analyz}.

\subsubsection{Критерий <<Цена>>}

В табл. \ref{cost} проведены попарные сравнения альтернатив по критерию A1 <<Цена>>.
\begin{table}[H]
  \caption{Матрица попарных сравнений по критерию A1}\label{cost}
  \begin{tabular}{|l|l|l|l|}
  \hline Альтернатива & M1 & M2 & M3 \\
  \hline M1 & 1 & 4 & 6 \\
  \hline M2 & 1/4 & 1 & 3 \\
  \hline M3 & 1/6 & 1/3 & 1 \\
  \hline
  \end{tabular}
\end{table}

Согласно формуле (\ref{geomean}) вычислим сравнительную желательность альтернатив по первому критерию:

$b_1 = \sqrt[3]{1 \cdot 4 \cdot 6} = \sqrt[3]{24} = 2,8845$

$b_2 = \sqrt[3]{0,25 \cdot 1 \cdot 3} = \sqrt[3]{0,75} = 0,9086$

$b_3 = \sqrt[3]{0,1667 \cdot 0,3333 \cdot 1} = \sqrt[3]{0,0556} = 0,3817$

Просуммируем полученные значения:

$B = 2,8845 + 0,9086 + 0,3817 = 4,1748$

Далее воспользуемся формулой (\ref{veccom}), заменив идентификаторы $x_i$ на $z_j$:

$z_1 = \frac{2,8845}{4,1748} = 0,6909$

$z_2 = \frac{0,9086}{4,1748} = 0,2176$

$z_3 = \frac{0,3817}{4,1748} = 0,0914$

Проведем проверку по формуле (\ref{vecsum}):

$\sum_{i=1}^{3} 0,6909 + 0,2176 + 0,0914 = 0,9999$

Оценка погрешности вычисляется по формуле (\ref{fault}):
\begin{equation}\label{fault}
\delta_{x} = \frac{|1 - \sum_{i=1}^{n} z_i|}{1} \cdot 100\%, i = \overline{1,n}
\end{equation}

Оценим погрешность вычислений:

$\delta_{x} = \frac{|1 - 0,9999|}{1} \cdot 100\% = 0,01$

Оценим отношение согласованности для матрицы попарных сравнений второго уровня по формуле (\ref{atcon}):

$y_1 = 1 + 0,25 + 0,1667 = 1,4167$

$y_2 = 4 + 1 + 0,3333 = 5,3333$

$y_3 = 6 + 3 + 1 = 10$

Вычислим наибольшее собственное значение матрицы сравнений согласно (\ref{maxm}):

$\lambda_{max} = 0,6909 \cdot 1,4167 + 0,2176 \cdot 5,3333 + 0,0914 \cdot 10 = 0,9788 + 1,1605 + 0,914 = 3,0533$

Вычислим индекс согласованности для данной матрицы:

$\text{ИС} = \frac{3,0533 - 3}{2} = 0,0267$

Далее найдем отношение согласованности:

$\text{ОС} = \frac{0,0267}{0,58} \cdot 100\% = 4,6\%$

Полученное значение $\text{ОС} = 4,6\% < 20\%$.
Считаем, что матрица парных сравнений третьего уровня по критерию А1 является согласованной.

\subsubsection{Критерий <<Масштабируемость>>}

В табл. \ref{agil} проведены попарные сравнения альтернатив по критерию A2 <<Масштабируемость>>.
\begin{table}[H]
  \caption{Матрица попарных сравнений по критерию A2}\label{agil}
  \begin{tabular}{|l|l|l|l|}
  \hline Альтернатива & M1 & M2 & M3 \\
  \hline M1 & 1 & 3 & 1 \\
  \hline M2 & 1/3 & 1 & 1/3 \\
  \hline M3 & 1 & 3 & 1 \\
  \hline
  \end{tabular}
\end{table}

Согласно формуле (\ref{geomean}) вычислим сравнительную желательность альтернатив по первому критерию:

$b_1 = \sqrt[3]{1 \cdot 3 \cdot 1} = \sqrt[3]{3} = 1,4422$

$b_2 = \sqrt[3]{0,3333 \cdot 1 \cdot 0,3333} = \sqrt[3]{0,1111} = 0,4807$

$b_3 = \sqrt[3]{1 \cdot 3 \cdot 1} = \sqrt[3]{3} = 1,4422$

Просуммируем полученные значения:

$B = 1,4422 + 0,4807 + 1,4422 = 3,3651$

Далее воспользуемся формулой (\ref{veccom}), заменив идентификаторы $x_i$ на $z_j$:

$z_1 = \frac{1,4422}{3,3651} = 0,4286$

$z_2 = \frac{0,4807}{3,3651} = 0,1429$

$z_3 = \frac{1,4422}{3,3651} = 0,4286$

Проведем проверку по формуле (\ref{vecsum}):

$\sum_{i=1}^{3} 0,4286 + 0,1429 + 0,4286 = 1,0001$

Оценим погрешность вычислений:

$\delta_{x} = \frac{|1 - 1,0001|}{1} \cdot 100\% = 0,01$

Оценим отношение согласованности для матрицы попарных сравнений второго уровня по формуле (\ref{atcon}):

$y_1 = 1 + 0,3333 + 1 = 2,3333$

$y_2 = 3 + 1 + 3 = 7$

$y_3 = 1 + 0,3333 + 1 = 2,3333$

Вычислим наибольшее собственное значение матрицы сравнений согласно (\ref{maxm}):

$\lambda_{max} = 0,4286 \cdot 2,3333 + 0,1429 \cdot 7 + 0,4286 \cdot 2,3333 = 1,0001 + 1,0003 + 1,0001 = 3,0005$

Вычислим индекс согласованности для данной матрицы:

$\text{ИС} = \frac{3,0005 - 3}{2} = 0,0003$

Далее найдем отношение согласованности:

$\text{ОС} = \frac{0,0003}{0,58} \cdot 100\% = 0,05\%$

Полученное значение $\text{ОС} = 0,05\% < 20\%$.
Считаем, что матрица парных сравнений третьего уровня по критерию А2 является согласованной.

\subsubsection{Критерий <<Отказоустойчивость>>}

В табл. \ref{fat} проведены попарные сравнения альтернатив по критерию A3 <<Отказоустойчивость>>.
\begin{table}[H]
  \caption{Матрица попарных сравнений по критерию A3}\label{fat}
  \begin{tabular}{|l|l|l|l|}
  \hline Альтернатива & M1 & M2 & M3 \\
  \hline M1 & 1 & 1/3 & 1/3 \\
  \hline M2 & 3 & 1 & 1/3 \\
  \hline M3 & 3 & 3 & 1 \\
  \hline
  \end{tabular}
\end{table}

Согласно формуле (\ref{geomean}) вычислим сравнительную желательность альтернатив по первому критерию:

$b_1 = \sqrt[3]{1 \cdot 0,3333 \cdot 0,3333} = \sqrt[3]{0,1111} = 0,4807$

$b_2 = \sqrt[3]{3 \cdot 1 \cdot 0,3333} = \sqrt[3]{0,9999} = 1$

$b_3 = \sqrt[3]{3 \cdot 3 \cdot 1} = \sqrt[3]{9} = 2,0801$

Просуммируем полученные значения:

$B = 0,4807 + 1 + 2,0801 = 3,5608$

Далее воспользуемся формулой (\ref{veccom}), заменив идентификаторы $x_i$ на $z_j$:

$z_1 = \frac{0,4807}{3,5608} = 0,135$

$z_2 = \frac{1}{3,5608} = 0,2808$

$z_3 = \frac{2,0801}{3,5608} = 0,5842$

Проведем проверку по формуле (\ref{vecsum}):

$\sum_{i=1}^{3} 0,135 + 0,2808 + 0,5842 = 1$

Оценим погрешность вычислений:

$\delta_{x} = \frac{|1 - 1|}{1} \cdot 100\% = 0$

Оценим отношение согласованности для матрицы попарных сравнений второго уровня по формуле (\ref{atcon}):

$y_1 = 1 + 3 + 3 = 7$

$y_2 = 0,3333 + 1 + 3 = 4,3333$

$y_3 = 0,3333 + 0,3333 + 1 = 1,6666$

Вычислим наибольшее собственное значение матрицы сравнений согласно (\ref{maxm}):

$\lambda_{max} = 0,135 \cdot 7 + 0,2808 \cdot 4,3333 + 0,5842 \cdot 1,6666 = 0,945 + 1,2168 + 0,9736 = 3,1354$

Вычислим индекс согласованности для данной матрицы:

$\text{ИС} = \frac{3,1354 - 3}{2} = 0,0677$

Далее найдем отношение согласованности:

$\text{ОС} = \frac{0,0677}{0,58} \cdot 100\% = 11,67\%$

Полученное значение $\text{ОС} = 11,67\% < 20\%$.
Считаем, что матрица парных сравнений третьего уровня по критерию А3 является согласованной.

\subsubsection{Критерий <<Интерфейсы управления>>}

В табл. \ref{interf} проведены попарные сравнения альтернатив по критерию A4 <<Интерфейсы управления>>.
\begin{table}[H]
  \caption{Матрица попарных сравнений по критерию A4}\label{interf}
  \begin{tabular}{|l|l|l|l|}
  \hline Альтернатива & M1 & M2 & M3 \\
  \hline M1 & 1 & 1/3 & 1/3 \\
  \hline M2 & 3 & 1 & 1/3 \\
  \hline M3 & 3 & 3 & 1 \\
  \hline
  \end{tabular}
\end{table}

Согласно формуле (\ref{geomean}) вычислим сравнительную желательность альтернатив по первому критерию:

$b_1 = \sqrt[3]{1 \cdot 0,3333 \cdot 0,3333} = \sqrt[3]{0,1111} = 0,4807$

$b_2 = \sqrt[3]{3 \cdot 1 \cdot 0,3333} = \sqrt[3]{0,9999} = 1$

$b_3 = \sqrt[3]{3 \cdot 3 \cdot 1} = \sqrt[3]{9} = 2,0801$

Просуммируем полученные значения:

$B = 0,4807 + 1 + 2,0801 = 3,5608$

Далее воспользуемся формулой (\ref{veccom}), заменив идентификаторы $x_i$ на $z_j$:

$z_1 = \frac{0,4807}{3,5608} = 0,135$

$z_2 = \frac{1}{3,5608} = 0,2808$

$z_3 = \frac{2,0801}{3,5608} = 0,5842$

Проведем проверку по формуле (\ref{vecsum}):

$\sum_{i=1}^{3} 0,135 + 0,2808 + 0,5842 = 1$

Оценим погрешность вычислений:

$\delta_{x} = \frac{|1 - 1|}{1} \cdot 100\% = 0$

Оценим отношение согласованности для матрицы попарных сравнений второго уровня по формуле (\ref{atcon}):

$y_1 = 1 + 3 + 3 = 7$

$y_2 = 0,3333 + 1 + 3 = 4,3333$

$y_3 = 0,3333 + 0,3333 + 1 = 1,6666$

Вычислим наибольшее собственное значение матрицы сравнений согласно (\ref{maxm}):

$\lambda_{max} = 0,135 \cdot 7 + 0,2808 \cdot 4,3333 + 0,5842 \cdot 1,6666 = 0,945 + 1,2168 + 0,9736 = 3,1354$

Вычислим индекс согласованности для данной матрицы:

$\text{ИС} = \frac{3,1354 - 3}{2} = 0,0677$

Далее найдем отношение согласованности:

$\text{ОС} = \frac{0,0677}{0,58} \cdot 100\% = 11,67\%$

Полученное значение $\text{ОС} = 11,67\% < 20\%$.
Считаем, что матрица парных сравнений третьего уровня по критерию А4 является согласованной.

\subsection{Анализ результатов оценки альтернатив}

По полученным значениям векторов локальных приоритетов сделаем выводы о важности альтернатив для каждого из критериев.

Критерий <<Цена>>:
\begin{enumerate}
  \item альтернатива А ($0,6909$);
  \item альтернатива Б ($0,2176$);
  \item альтернатива В ($0,0914$).
\end{enumerate}

Альтернатива А имеет явное преимущество над альтернативой Б, которая в свою очередь имеет явное преимущество над альтернативой В.

Критерий <<Масштабируемость>>:
\begin{enumerate}
  \item альтернатива А ($0,4286$);
  \item альтернатива Б ($0,1429$);
  \item альтернатива В ($0,4286$).
\end{enumerate}

Альтернативы А и В имеют равные преимущества над альтернативой Б.

Критерий <<Отказоустойчивость>>:
\begin{enumerate}
  \item альтернатива А ($0,135$);
  \item альтернатива Б ($0,2808$);
  \item альтернатива В ($0,5842$).
\end{enumerate}

Альтернатива В имеет преимущество над альтернативой Б, которая в свою очередь имеет преимущество перед альтернативой А.

Критерий <<Интерфейсы управления>>:
\begin{enumerate}
  \item альтернатива А ($0,135$);
  \item альтернатива Б ($0,2808$);
  \item альтернатива В ($0,5842$).
\end{enumerate}

Альтернатива В имеет преимущество над альтернативой Б, которая в свою очередь имеет преимущество перед альтернативой А.

Из полученных данных, можно сделать вывод, что альтернативы имеют равное значение и отдать небольшое предпочтение можно альтернативам А и В по сравнению с альтернативой Б.

\subsection{Синтез глобальных приоритетов альтернатив}

Для выявления составных, или глобальных, приоритетов методов решения нелинейных уравнений, локальные приоритеты альтернатив располагаются по отношению к каждому критерию.
Каждый столбец векторов альтернатив умножается на приоритет соответствующего критерия и результаты складываются вдоль каждой строки \cite{var-analyz}, формула (\ref{globpr}):
\begin{equation}\label{globpr}
V^{M_j} = \sum_{i = 1}^{n} x_i \cdot z_{i}^{M_j}
\end{equation}

В табл. \ref{globdata} для наглядности представлены исходные данные для расчета значения компонент вектора глобальных приоритетов, полученные на предыдущих этапах МАИ.
\begin{table}[H]
  \caption{Данные для расчета глобальных приоритетов}\label{globdata}
  \begin{tabular}{|l|l|l|l|l|}
  \hline  & $x_1 (0,0828)$ & $x_2 (0,2489)$ & $x_3 (0,6259)$ & $x_4 (0,0426)$ \\
  \hline $z_{i}^{M_1}$ & $0,6909$ & $0,4286$ & $0,135$ & $0,135$ \\
  \hline $z_{i}^{M_2}$ & $0,2176$ & $0,1429$ & $0,2808$ & $0,2808$ \\
  \hline $z_{i}^{M_3}$ & $0,0914$ & $0,4286$ & $0,5842$ & $0,5842$ \\
  \hline
  \end{tabular}
\end{table}

Рассчитаем по формуле (\ref{globpr}):

$V^{M_1} = 0,0828 \cdot 0,6909 + 0,2489 \cdot 0,4286 + 0,6259 \cdot 0,135 + 0,0426 \cdot 0,135 = 0,0572 + 0,1067 + 0,0845 + 0,0058 = 0,2542$

$V^{M_2} = 0,0828 \cdot 0,2176 + 0,2489 \cdot 0,1429 + 0,6259 \cdot 0,2808 + 0,0426 \cdot 0,2808 = 0,018 + 0,0356 + 0,1758 + 0,012 = 0,2414$

$V^{M_2} = 0,0828 \cdot 0,0914 + 0,2489 \cdot 0,4286 + 0,6259 \cdot 0,5842 + 0,0426 \cdot 0,5842 = 0,018 + 0,1067 + 0,3657 + 0,0249 = 0,5135$

Для проверки вычислений воспользуемся формулой (\ref{globsum}):
\begin{equation}\label{globsum}
\sum_{i = 1}^{n} V^{M_j} = 1
\end{equation}

$\sum_{i = 1}^{3} V^{M_j} = 0,2542 + 0,2414 + 0,5135 = 1,0091$

Рассчитаем погрешность вычислений:

$\delta_{V} = \frac{1 - 1,0091}{3} \cdot 100\% = 0,3\%$

Полученная погрешность не превышает допустимые $20\%$, следовательно, будем считать погрешность удовлетворительной.

Чтобы оценить согласованность всей иерархии, необходимо воспользоваться формулой (\ref{osh}):
\begin{equation}\label{osh}
\text{ОС}_{\text{Иерархии}} = \frac{\sum_{i = 1}^{n} x_i \cdot \text{ИС}_i}{\text{СС}(m)} \cdot 100\%
\end{equation}
где $x_i$ --- значение i-ой компоненты вектора локальных приоритетов второго уровня.
$\text{ИС}_i$ --- значение i-го индекса согласованности матриц попарных сравнений третьего уровня.
$\text{СС}(m)$ --- значение случайной согласованности для $m = 3$ \cite{var-analyz}.

Выполним расчет:

$\text{ОС}_{\text{Иерархии}} = \frac{0,0828 \cdot 0,0267 + 0,2489 \cdot 0,0003 + 0,6259 \cdot 0,0677 + 0,0426 \cdot 0,0677}{0,58} \cdot 100\% = \frac{0,0022 + 0,0001 + 0,0423 + 0,0029}{0,58} \cdot 100\% = \frac{0,0475}{0,58} \cdot 100\% = 8,19\%$

Полученное значение допустимо так как, не превышает $20\%$.

Результаты вычислений по формуле (\ref{globpr}) можно рассматривать как значения функции полезности для каждой из альтернатив \cite{var-analyz}.
Теперь можно оценивать альтернативы по убыванию значений функции полезности.

На первом месте альтернатива В (VMware vSphere), которая <<выигрывает>> у альтернативы А (KVM):

$(0,5135 - 0,2542) \cdot 100\% = 25,93\%$

Альтернатива А <<выигрывает>> у альтернативы Б (Hyper-V):

$(0,2542 - 0,2414) \cdot 100\% = 1,28\%$

Далее вычислим вклады критериев в окончательные результаты.
Для вычисления $q_{i}^{M_j}$ (<<вклада>> i-го критерия в значение функции полезности $V^{M_j}$) воспользуемся значениями промежуточных результатов в формуле (\ref{globpr}) и подставим их в формулу (\ref{contr}) \cite{var-analyz}:
\begin{equation}\label{contr}
q_{i}^{M_j} = \frac{x_i \cdot z_{i}^{M_j}}{V^{M_j}} \cdot 100\%, i = \overline{1, 4}, j = \overline{1, 3}
\end{equation}

Определим количественные оценки в процентном соотношении вклада каждого из критериев подробнее и сделаем выводы.

Альтернатива А (KVM):

$q_{1}^{M_1} = \frac{0,0572}{0,2542} \cdot 100\% = 22,5\%$

$q_{2}^{M_1} = \frac{0,1067}{0,2542} \cdot 100\% = 44,97\%$

$q_{3}^{M_1} = \frac{0,0845}{0,2542} \cdot 100\% = 33,24\%$

$q_{4}^{M_1} = \frac{0,0058}{0,2542} \cdot 100\% = 2,28\%$

Исходя из полученных результатов, делаем вывод, что критерий <<Масштабируемость>> внес наибольший вклад в результат данной альтернативы, так как имеет наивысший приоритет.
Так же значительный вес внесли критерии <<Отказоустойчивость>> и <<Цена>>.
Критерий <<Интерфейсы управления>> не оказал сильного влияния на результат.

Альтернатива Б (Hyper-V):

$q_{1}^{M_2} = \frac{0,018}{0,2414} \cdot 100\% = 7,45\%$

$q_{2}^{M_2} = \frac{0,0356}{0,2414} \cdot 100\% = 14,75\%$

$q_{3}^{M_2} = \frac{0,1758}{0,2414} \cdot 100\% = 72,83\%$

$q_{4}^{M_2} = \frac{0,012}{0,2414} \cdot 100\% = 4,97\%$

Критерий <<Отказоустойчивость>> внес наибольший вклад в результат данной альтернативы.
Остальные критерий не оказали сильного влияния на результат.

Альтернатива В (VMware vSphere):

$q_{1}^{M_3} = \frac{0,018}{0,5135} \cdot 100\% = 3,51\%$

$q_{2}^{M_3} = \frac{0,1067}{0,5135} \cdot 100\% = 20,78\%$

$q_{3}^{M_3} = \frac{0,3657}{0,5135} \cdot 100\% = 71,22\%$

$q_{4}^{M_3} = \frac{0,0249}{0,5135} \cdot 100\% = 4,85\%$

Критерий <<Отказоустойчивость>> внес наибольший вклад в результат данной альтернативы.
Критерий <<Масштабируемость>> оказал небольшое влияние на результат.
Остальные критерий не оказали сильного влияния.

\subsection{Анализ результатов}

Представим все полученные результаты вкладов критериев в виде табл. \ref{critdata}.
\begin{table}[H]
  \caption{Вклады критериев}\label{critdata}
  \begin{tabular}{|l|l|l|l|l|}
  \hline  & A1 & A2 & A3 & A4 \\
  \hline $q_{i}^{M_1}$ & $22,5\%$ & $44,97\%$ & $33,24\%$ & $2,28\%$ \\
  \hline $q_{i}^{M_2}$ & $7,45\%$ & $14,75\%$ & $72,83\%$ & $4,97\%$ \\
  \hline $q_{i}^{M_3}$ & $3,51\%$ & $20,78\%$ & $71,22\%$ & $4,85\%$ \\
  \hline
  \end{tabular}
\end{table}

В результате проведения вариантного анализа определили, что взяв во внимание значения функции полезности для каждой из альтернатив и вклады каждого критерия, наибольшее предпочтение в выборе следует отдать альтернативе В (VMware vSphere), которая <<выигрывает>> у других альтернатив на $25,93\%$ и $27,21\%$.

Также стоит упомянуть, что при выборе гипервизора на практике, недостаточно руководствоваться лишь несколькими критериями, так как для каждого конкретного случая необходимо рассматривать довольно обширный список критериев.

\clearpage
