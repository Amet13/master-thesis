\anonsection{Заключение}

В ходе выполнения выпускной квалификационной работы магистра были исследованы процессы обеспечения безопасности облачных сред.

В ходе исследования были проанализированы существующие проблемы и стандарты безопасности облачных вычислений, предложены способы решения данных проблем.
Рассмотрена специфика предоставления облачных услуг зарубежных и отечественных поставщиков.
Проанализированы наиболее опасные уязвимости за 2016~г.

В ходе системного анализа было описано системотехническое представление системы, описаны входные и выходные данные, составлен список функций системы безопасности, произведена декомпозиция системы и описана связь между ее элементами.

В ходе вариантного анализа был произведен сравнительный анализ гипервизоров между тремя альтернативами, в ходе которого был выбран наиболее оптимальный вариант.

В ходе экспериментальных исследования была эксплуатирована уязвимость CVE-2016-5195, благодаря которой локальный пользователь сервера получил доступ к правам суперпользователя.
Скрипт не является законченным продуктом и распространяется под свободной лицензией.

Для мониторинга уязвимостей в программном обеспечении облачной среды был разработан скрипт на языке программирования Python.
Скрипт осуществляет поиск по открытой базе уязвимостей согласно установленным параметрам.
Программа находится в открытом доступе и распространяется под открытой лицензией.

Практическая значимость исследования состоит в возможности применения написанной программы в облачной среде провайдеров для анализа уязвимостей и незамедлительного реагирования на них.
Также разработана защищенная облачная инфраструктура, по примеру которой можно аналогично строить другие, описаны страгегии расширения инфраструктуры.

\clearpage
