\section{Постановка задачи}

Конечной задачей выпускной квалификационной работы магистра на тему <<Исследование процессов обеспечения безопасности облачных сред>> является подробный анализ стандартов безопасности облачных вычислений, варианты решения данных проблем, а также технические возможности практической эксплуатации уязвимостей на нескольких уровнях работы облачной инфраструктуры.

Исследования должны состоять из следующих частей:
\begin{itemize}
  \item обзор составных частей облачной инфраструктуры;
  \item анализ технологий используемых облачными провайдерами, необходимых для построения облачной инфраструктуры;
  \item специфика применений облачных вычислений в России;
  \item исследование проблемы безопасности облачных вычислений;
  \item решение проблем безопасности облаков;
  \item практическое применение уязвимостей в облачной среде, с использованием программ, распространяющихся под свободными лицензиями, например \hyperlink{gnu}{GNU} \hyperlink{gpl}{GPL}.
\end{itemize}

Для применения практических навыков исследования уязвимостей необходима аппаратная платформа со следующими характеристиками:
\begin{itemize}
  \item процессор Intel Core\textregistered~i3 2.3 ГГц с поддержкой аппаратной виртуализации;
  \item минимальный объем \hyperlink{ram}{ОЗУ} 8 Гб, рекомендуемый --- не менее 10 Гб;
  \item минимум 15 Гб места на жестком диске (\hyperlink{ssd}{SSD});
  \item операционная система Ubuntu 16.04, CentOS 7 или Debian 8 GNU/Linux.
\end{itemize}

Данная задача также рассматривается с точки зрения системного и вариантного анализа.

Системный анализ включает в себя \cite{sys-analyz}:
\begin{itemize}
  \item системотехническое представление системы безопасности в виде <<черного ящика>>;
  \item описание входных и выходных данных;
  \item список функций, которые выполняет система безопасности;
  \item учет случайностей;
  \item декомпозицию системы и описание связей между ее элементами.
\end{itemize}

Вариантный анализ произведен исходя выбранных критериев \cite{var-analyz}:
\begin{itemize}
  \item раз;
  \item два;
  \item три;
  \item ...;
  \item последний критерий.
\end{itemize}

\clearpage
