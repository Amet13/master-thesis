\begingroup
\renewcommand{\section}[2]{\anonsection{Библиографический список}}
\begin{thebibliography}{00}

\bibitem{telecom-world}
    Прудникова, А.А.
    Безопасность облачных вычислений /
    А.А. Прудникова //
    Мир телекома. -- 2013. -- №1. -- С. 50-55.

\bibitem{sys-analyz}
    Методические указания <<Процедура системного анализа при проектировании программных систем>>
    для студентов-дипломников дневной и заочной формы обучения специальности 7.091501 /
    Сост.: Сергеев Г.Г., Скатков А.В., Мащенко Е.Н. -- Севастополь:
    Изд-во СевНТУ, 2005. -- 32 с.

\bibitem{var-analyz}
    Методические указания к расчетно-графическому заданию
    на тему <<Метод анализа иерархий>>  по дисциплине <<Теория оптимальных решений>>
    для студентов специальности 7.091501 <<Компьютерные системы и сети>>
    дневной и заочной формы обучения /
    Сост.: Ю.Н. Щепин -- Севастополь:
    Изд-во СевНТУ, 2008. -- 28 с.

\bibitem{mai}
    Блюмин С.Л., Шуйкова И.А.
    Модели и методы принятия решений в условиях неопределенности. --
    Липецк: ЛЭГИ, 2001. -- 138 с.

\bibitem{nist}
    Hogan, M.
    NIST Cloud Computing Standarts Roadmap /
    M. Hogan, F. Liu, A. Sokol, J. Tong //
    NIST Special Publication 500-291, Version 2
    Roadmap Working Group, 2013. -- 113 с.

\bibitem{gemalto}
    The 2016 Global Cloud Data Security Study.
    Ponemon Insitute LLC, 2016. -- 40 с.

\bibitem{itmo}
    Беккер, М.Я.
    Информационная безопасность при облачных вычислениях: проблемы и перспективы /
    М.Я. Беккер, Ю.А. Гатчин, Н.С. Кармановский, А.О. Терентьев, Д.Ю. Федоров //
    Научно-технический вестник Санкт-Петербургского государственного университета информационных технологий, механики и оптики. -- 2011. -- №1(71). -- С. 97-102.

\bibitem{psta}
    Емельянова, Ю.Г.
    Анализ проблем и перспективы создания интеллектуальной системы обнаружения и предотвращения сетевых атак на облачные вычисления /
    Ю.Г. Емельянова, В.П. Фраленко //
    Программные системы: теория и приложения. -- 2011. -- №4(8) -- С. 17-31.

\bibitem{xen}
    Chisnall, D.
    The Definitive Guide to the Xen Hypervisor /
    D. Chisnall. --
    1st Edition //
    Prentice Hall Open Source Software Development, 2007. -- 320 с.

\bibitem{minsvyaz}
    Федеральный закон от 21 июля 2014 г. № 242-ФЗ <<О внесении изменений в отдельные законодательные акты Российской Федерации в части уточнения порядка обработки персональных данных в информационно-телекоммуникационных сетях>> /
    Минкомсвязь России //
    Опубликован 12.02.2016 на официальном интернет-портале Министерства связи и массовых коммуникаций Российской Федерации

\bibitem{cnews}
    Облачные сервисы 2016
    [Электронный ресурс] //
    CNews Analytics
    Режим доступа: https://goo.gl/cmDSMB
    (Дата обращения: 30.12.2016)

\bibitem{csa}
    Cloud Security Alliance Releases 'The Treacherous Twelve' Cloud Computing Top Threats in 2016
    [Электронный ресурс] //
    Cloud Security Alliance Research Group
    Режим доступа: https://goo.gl/l2aWLu
    (Дата обращения: 11.01.2017)

\bibitem{cnewsdc}
    ИТ-инфраструктура предприятия 2010: Пути оптимизации
    [Электронный ресурс] //
    CNews Analytics
    Режим доступа: https://goo.gl/jzrrIO
    (Дата обращения: 05.01.2017)

\bibitem{greendc}
    Kaplan, J.
    Revolutionizing data center energy efficiency /
    J. Kaplan, W. Forrest, N. Kindler //
    Technical report, McKinsey \& Company, 2008. -- 15 с.

\bibitem{aws}
    AWS signature version 1 is insecure
    [Электронный ресурс] //
    Daemonic Dispatches
    Режим доступа: https://goo.gl/70bggH
    (Дата обращения: 08.02.2017)

\bibitem{cis}
    The CIS Critical Security Controls for Effective Cyber Defense
    [Электронный ресурс] //
    SANS website
    Режим доступа: https://goo.gl/pMjbNE
    (Дата обращения: 08.02.2017)

\bibitem{owasp}
    OWASP Top Ten Project
    [Электронный ресурс] //
    OWASP website
    Режим доступа: https://goo.gl/kSHOjF
    (Дата обращения: 08.02.2017)

\bibitem{cvedetails}
    CVE security vulnerability database. Security vulnerabilities, exploits, references and more
    [Электронный ресурс] //
    CVE Details. The ultimate security vulnerability datasource
    Режим доступа: https://goo.gl/I3RtO2
    (Дата обращения: 20.02.2017)

\bibitem{dcow}
    Dirty COW (CVE-2016-5195) is a privilege escalation vulnerability in the Linux Kernel
    [Электронный ресурс] //
    CVE-2016-5195 info website
    Режим доступа: https://goo.gl/ziy3Nd
    (Дата обращения: 20.02.2017)

\bibitem{xsa182}
    Bug 1355987 - (CVE-2016-6258, xsa182) CVE-2016-6258 xsa182 xen: x86: Privilege escalation in PV guests (XSA-182)
    [Электронный ресурс] //
    Red Hat Bugzilla
    Режим доступа: https://goo.gl/dlqtnR
    (Дата обращения: 20.02.2017)

\bibitem{tcp}
    CVE-2016-5696
    [Электронный ресурс] //
    Common Vulnerabilities and Exposures. The Standart for Information Security Vulnerability Names
    Режим доступа: https://goo.gl/xYpFQQ
    (Дата обращения: 21.02.2017)

\bibitem{qemu}
    CVE-2016-5696
    [Электронный ресурс] //
    Debian Security Bug Tracker
    Режим доступа: https://goo.gl/BXkTiL
    (Дата обращения: 21.02.2017)

\bibitem{netraw}
    CVE-2016-8655 - Red Hat Customer Portal
    [Электронный ресурс] //
    Red Hat Customer Portal
    Режим доступа: https://goo.gl/QhVbmm
    (Дата обращения: 21.02.2017)

\bibitem{netf}
    CVE-2016-4997
    [Электронный ресурс] //
    Common Vulnerabilities and Exposures. The Standart for Information Security Vulnerability Names
    Режим доступа: https://goo.gl/dbtXny
    (Дата обращения: 21.02.2017)

\bibitem{cryptsetup}
    CVE-2016-4484: Cryptsetup Initrd root Shell
    [Электронный ресурс] //
    Hector Marco Gisbert - Lecturer and Cyber Security Researcher website
    Режим доступа: https://goo.gl/Jrfg6H
    (Дата обращения: 22.02.2017)

\bibitem{ecryptfs}
    CVE-2016-1583
    [Электронный ресурс] //
    Debian Security Bug Tracker
    Режим доступа: https://goo.gl/PIdqGR
    (Дата обращения: 22.02.2017)

\bibitem{dcowexp}
    gbonacini/CVE-2016-5195: A CVE-2016-5195 exploit example.
    [Электронный ресурс] //
    GitHub
    Режим доступа: https://goo.gl/9tFhHh
    (Дата обращения: 24.02.2017)

\end{thebibliography}
\endgroup

\clearpage
