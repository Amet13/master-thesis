\anonsection{Аннотация}


---

Актуальность: облака везде, облака нужны всем, не только бизнес-клиентам, но и обычным людям.

Т.к. популярность облаков появилась сравнительно недавно и она стремительно развивается, не всегда успевают учесть все аспекты безопасности.

Также из-за того, что облако состоит из большого количества ПО на различных уровнях, нужно учитывать все уязвимости, так как они могут всплыть на каждом из уровней.

Мысли:

* клиенты слабо представляют насколько защищены облака, поэтому предпочитают частное облаков, взамен публичного, не доверяют провайдеру

* основные аспекты облаков: мониторинг, управление, безопасность, доступность

* если надо добавить часть по экономике - файл 124.pdf

* по поводу иаас, преимущества понятны, а вот минусы в том, что если получают доступ к хост-ноде, то все, также это может быть изнутри, например уязвимость на гипервизоре

* конкретные проблемы с табличками описаны тут 1608.08787v1.pdf

* файл cloud-security-study-report.pdf конкретный отчет сравнение использования облаков в 2016 году по сравнению с 2014

* в файле informatsionnaya-bezopasnost-pri-oblachnyh-vychisleniyah-problemy-i-perspektivy.pdf хорошо расписан вопрос по стандартизации облаков, также кратко написано про риски использования облаков

* тут тоже про стандарты psta2011-4-17-31.pdf

* в файле str50.pdf рассказывакется про проблемы в России, проблемы с точки зрения инф. безопасности

* реальные опросы от интела 2012г, которые рассказывают, что препятствуют уходу в облака, файл whats-holding-back-the-cloud-peer-research-report.pdf

* хорошая презентация Zegzhda-PD-supernova-2.pdf краткие тезисы, примеры картинок, модель безопасности даже есть

---

В данной выпускной квалификационной работе рассмотрены ...

Ключевые слова: 1, 2, 3, 4, 5.

Актуальность темы. Блаблабла.

Конечная цель проектирования тратата.

Выпускная квалификационная работа магистра изложена на 100500 листах, включает 1000 таблиц, 200 рисунков, 1 приложение, 19 литературных источников.

\clearpage
